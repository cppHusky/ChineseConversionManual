\section{\content{注音符號、漢語拼音簡介}{汉语拼音、注音符号简介}}\label{section:bopomofo-pinyin}
\subsection{\content{背景知識}{背景知识}}
\content{回想一下,在你幼時初學漢語(如果你的母語是漢語)的過程中,你是否有過「能認得某個字,但就是不會寫」的經歷?你也可以問問身邊常用漢語的朋友們,他們是否常常提筆忘字?}{回想一下,在你幼时学习汉语(如果你的母语是汉语)的过程中,你是否有过“能认得某个字,但就是不会写”的经历?你也可以问问身边常用汉语的朋友们,他们是否常常提笔忘字?}\par
\content{這樣的情況並不鮮見,因為語言習得的正常規律是從「聽」和「說」開始的。當我們開始學寫字時,其實我們早就會唸不少字了;每學一個新字時,我們也總會更關注它的讀音,而不是寫法。}{这样的情况并不鲜见,因为语言习得的正常规律是从“听”和“说”开始的。当我们开始学写字时,其实我们早就会念不少字了;每学一个新字时,我们也总会更关注它的读音,而不是写法。}\par
\content{我們判斷自己是否「認識」某個字,往往也要以「是否會唸」作為依據。舉些例子,「\textbf{柃}木」「\textbf{囿}於」「虎\textbf{兕}」「\textbf{朊}粒」「\textbf{旻}天」中的五個粗體字,簡單到只要让人看過一眼就會寫了,但是你真的認得它們嗎?與之相反,對於「食\textbf{鹽}」「\textbf{黎}明」「\textbf{罐}頭」「\textbf{贏}家」中的四个粗體字,我們未必都會寫(或許正是人們提筆忘字的範例),但是我們只要看上一眼就可以確信,我們認識這些字!}{我们判断自己是否“认识”某个字,往往也要以“是否会念”作为依据。举些例子,“\textbf{柃}木”“\textbf{囿}于”“虎\textbf{兕}”“\textbf{朊}粒”“\textbf{旻}天”中的五个粗体字,简单到只要让人看过一眼就会写了,但是你真的认得它们吗?与之相反,对于“扫\textbf{帚}”“\textbf{黎}明”“\textbf{罐}头”“\textbf{赢}家”中的四个粗体字,我们未必都会写(或许正是人们提笔忘字的范例),但是我们只要看上一眼就可以确信,我们认识这些字!}\par
\content{可見,我們對「字音」的印象要比對「字形」深得多,所以在學習任何字時,掌握其讀音都是很有必要的。就書面注音而言,現代人一般使用注音符號或漢語拼音來為漢字注音,而古人則使用反切等方法為漢字注音。本節將對漢語注音方法的發展做簡要介紹。}{可见,我们对“字音”的印象要比对“字形”深得多,所以在学习任何字时,掌握其读音都是很有必要的。就书面注音而言,现代人一般使用汉语拼音或注音符号来为汉字注音,而古人则使用反切等方法为汉字注音。本节将对汉语注音方法的发展做简要介绍。}\par
\subsubsection{\content{同音相注法}{同音相注法}}
\content{試想,如果不能用注音/拼音這樣的記號,我們要怎麼表示一個字的讀音呢?最簡單的方法就是用另一個讀音相同的字來為它注音,這就是同音相注法,或稱直音法。}{试想,如果不能用拼音/注音这样的记号,我们要怎么表示一个字的读音呢?最简单的方法就是用另一个读音相同的字来为它注音,这就是同音相注法,或称直音法。}\par
\begin{wrapfigure}{O}{.5\textwidth}
    \centering
    \includegraphics[width=.5\textwidth]{新華字典中的同音相注法.png}
    \caption{\content{新華字典中的音節索引}{新华字典中的音节索引}}
    \label{figure:新華字典中的同音相注法}
    \footnotesize{\color{darkgray}\content{索引中保留了同音相注法}{索引中保留了同音相注法}}
\end{wrapfigure}
\content{舉個簡單的例子,如果要為「庵」字注音,我們可以用「安」字。這樣一來,稍有讀音知識的人都可以透過「安」字得知「庵」字的讀音。}{举个简单的例子,如果要为“庵”字注音,我们可以用“安”字。这样一来,稍有读音知识的人都可以通过“安”字得知“庵”字的读音。}\par
\content{《新華字典》的音節索引頁中既有拼音,又有對應拼音的例字(\cref*{figure:新華字典中的同音相注法})。我們可以認為這是某種意義上的「同音相注」:比如,與「安」同音的字在第三頁。}{《新华字典》的音节索引页中既有拼音,又有对应拼音的例字(\cref*{figure:新華字典中的同音相注法})。我们可以认为这是某种意义上的“同音相注”:比如,与“安”同音的字在第三页。}\par
\content{但是,同音相注法的缺陷也是顯而易見的:如果一個人連常用的漢字都認不全,那他就很難透過已有的讀音知識來判斷新漢字的讀音了。}{但是,同音相注法的缺陷也是显而易见的:如果一个人连常用的汉字都认不全,那他就很难通过已有的读音知识来判断新汉字的读音了。}\par
\content{同音相注法還有其它的尷尬之處。仍以\cref*{figure:新華字典中的同音相注法}為例,我們看「岑」字,這並不是一個常用字,很多人不認得它。然而,同讀音\bopin{ㄘㄣ}{cen}之下已沒有更常用的字,我們要想同音相注的話,根本找不到合適的選擇!}{同音相注法还有其它的尴尬之处。仍以\cref*{figure:新華字典中的同音相注法}为例,我们看“岑”字,这并不是一个常用字,很多人也不认得它。然而,同读音\bopin{ㄘㄣ}{cen}之下已没有更常用的字,我们要想同音相注的话,根本找不到合适的选择!}\par
\subsubsection{反切法}
\content{反\textbf{切}\textsuperscript{千介切}法在古漢語注音系統中有著舉足輕重的地位。它的基本思路是「二字注一字」:取第一字(上字)的聲母和第二字(下字)的韻母及音調\footnote{這樣說不太準確。關於音調,準確的說法是「陰陽判於上字,平仄斷乎下字」,即上下二字共同決定音調。然而,古今音調規則相差很大,本書無意在此處細究,故正文不提。},拼合起來就是所求之字(歸字)的讀音。}{反\textbf{切}\textsuperscript{千介切}法在古汉语注音系统中有着举足轻重的地位。它的基本思路是“二字注一字”:取第一字(上字)的声母和第二字(下字)的韵母及音调\footnote{这样说不太准确。关于音调,准确的说法是“阴阳判于上字,平仄断乎下字”,即上下二字共同决定音调。然而,古今音调规则相差很大,本书无意在此处细究,故正文不提。},拼合起来就是所求之字(归字)的读音。}\par
\begin{wrapfigure}{O}{.4\textwidth}
    \centering
    \vspace{-2em}
    \includegraphics[width=.4\textwidth]{說文-艸部.jpeg}
    \caption{\content{大徐本《說文》草部}{大徐本《说文》草部}}
    \label{figure:說文-艸部}
    \footnotesize{\color{darkgray}\content{圖片來源:維基共享資源}{图片来源:维基共享资源}}
\end{wrapfigure}
\content{方才文中標粗的「切」字是用「千」「介」二字來注音的,我們在拼讀「切」字時,只需取上字「千」的聲母\bopin{ㄑ}{q}和下字「介」的韻調\bopin{ㄧㄝˋ}{\pinyin{ie4}}就可以拼出「切」的字音\bopin{ㄑㄧㄝˋ}{\pinyin{qie4}}了。}{方才文中标粗的“切”字是用“千”“介”二字来注音的,我们在拼读“切”字时,只需取上字“千”的声母\bopin{ㄑ}{q}和下字“介”的韵调\bopin{ㄧㄝˋ}{\pinyin{ie4}}就可以拼出“切”的字音\bopin{ㄑㄧㄝˋ}{\pinyin{qie4}}了。}\par
\content{反切注音比同音相注靈活得多,比如「岑」字可以用「才」「人」二字切出,這樣就能在一定程度上避免同音相注法「無字可注」的尷尬。}{反切注音法比同音相注灵活得多,比如“岑”字可以用“才”“人”二字切出,这样就能在一定程度上避免同音相注法“无字可注”的尴尬。}\par
\content{更重要的是,反切對於語音系統的分析更為透徹。從反切開始,人們已經有了完善的聲韻觀念和系統化的拼讀規則。一個人只需認得一些常見的反切上下字,就可以拼讀任何注音文字。東漢以後,反切法大行其道,無論韻書、辭典,無一不用反切注音(\cref*{figure:說文-艸部})。}{更重要的是,反切对于语音系统的分析更为透彻。从反切开始,人们已经有了完善的声韵观念和系统化的拼读规则。一个人只需认得一些常见的反切上下字,就可以拼读任何注音文字。东汉以后,反切法大行其道,无论韵书、辞典,无一不用反切注音(\cref*{figure:說文-艸部})。}\par
\subsubsection{\content{威妥瑪拼音}{威妥玛拼音}(Wei\textsuperscript{1}\ T'o\textsuperscript{3}-ma\textsuperscript{3}\ P'in\textsuperscript{1}-yin\textsuperscript{1})}
\content{威妥瑪拼音產生於19世紀後半葉,它是使用拉丁字母為漢語注音的較早嘗試,也是19\textasciitilde20世紀期間使用最為廣泛的拉丁注音。}{威妥玛拼音产生于19世纪后半叶,它是使用拉丁字母为汉语注音的较早尝试,也是19\textasciitilde20世纪期间使用最为广泛的拉丁注音。}\par
\content{在漢語拼音出現後,威妥瑪拼音已經逐漸被漢語拼音所取代。但是,我們仍然可以在部分英譯名中看到威妥瑪拼音的痕跡,比如「北京大學」是「\textbf{Peking} University」,「台北」是「\textbf{Taipei}」,「太極」是「\textbf{Taichi}」,「易經」是「{\bf I Ching}」。這些都是威妥瑪拼音的拼法。}{在汉语拼音出现后,威妥玛拼音已经逐渐被汉语拼音所取代。但是,我们仍然可以在部分英译名中看到威妥玛拼音的痕迹,比如“北京大学”是“\textbf{Peking} University”,“台北”是“\textbf{Taipei}”,“太极”是“\textbf{Taichi}”,“易经”是“{\bf I Ching}”。这些都是威妥玛拼音的拼法。}\par
\content{威妥瑪拼音是用來拼讀現代標準漢語的,它與注音符號、漢語拼音之間有較好的對應關係。威妥瑪拼音的使用時間相當長,直到2008年,台灣在仍廣泛使用。}{威妥玛拼音是用来拼读现代标准汉语的,它与注间符号、汉语拼音之间有较好的对应关系。威妥玛拼音的使用时间相当长,直到2008年,台湾仍在广泛使用。}\par
\subsubsection{\content{\ruby[m]{注}{ㄓㄨˋ}\ruby[m]{音}{ㄧㄣ}\ruby[m]{符}{ㄈㄨˊ}\ruby[m]{號}{ㄏㄠˋ}}{\ruby[m]{注}{ㄓㄨˋ}\ruby[m]{音}{ㄧㄣ}\ruby[m]{符}{ㄈㄨˊ}\ruby[m]{号}{ㄏㄠˋ}}}
\begin{wrapfigure}{I}{.4\textwidth}
    \centering
    \includegraphics[width=.4\textwidth]{章太炎《駁中國用萬國新語說》「紐文」一頁.png}
    \caption{\content{章太炎《駁中國用萬國新語說》中的一頁,介紹的是紐文}{章太炎《驳中国用万国新语说》中的一页,介绍的是纽文}}
    \label{figure:章太炎《駁中國用萬國新語說》「紐文」一頁}
    \footnotesize{\color{darkgray}\content{《民報》第21號}{《民报》第21号}}
\end{wrapfigure}
\content{回顧一下前面講到的反切法吧。它的思路是很清晰的:一「聲」一「韻」得一字。但在實際操作中,反切法是用另兩個字分別提供的聲母和韻母來為某個字注音的。按維基大典的說法\footnote{\href{https://zh-classical.wikipedia.org/wiki/反切\#圓缺}{反切\#圓缺 - 維基大典}},反切法需要約四百個上字,約一千個下字,這就要求反切法的使用者先得認識許多字的讀音。而在使用反切法的過程中,我們還需要把上字的聲和下字的韻調分別提取出來再拼合,這個過程也是十分繁瑣的。}{回顾一下前面讲到的反切法吧。它的思路是很清晰的:一“声”一“韵”得一字。但在实际操作中,反切法是用另两个字分别提供的声母和韵母来为某个字注音的。按维基大典的说法\footnote{\href{https://zh-classical.wikipedia.org/wiki/反切\#圓缺}{反切\#圓缺 - 維基大典}},反切法需要约四百个上字,约一千个下字,这就要求反切法的使用者先得认识许多字的读音。而在使用反切法的过程中,我们还需要把上字的声和下字的韵调分别提取出来再拼合,这个过程也是十分繁琐的。}\par
\content{如果我們要重新設計一套拼讀系統的話,自然不難想到,其實我們完全可以自訂一套聲母和韻母,根本沒必要用別的漢字來「反切」聲韻。這也正是清末許多學者努力的方向。}{如果我们要重新设计一套拼读系统的话,自然不难想到,其实我们完全可以自定义一套声母和韵母,根本没必要用别的汉字来“反切”声韵。这也正是清末许多学者努力的方向。}\par
\content{章太炎模仿日語假名的做法,借用漢字的偏旁等部件創造了若干「紐文」「韻文」,這是注音符號的前身(\cref*{figure:章太炎《駁中國用萬國新語說》「紐文」一頁})。中華民國教育部在章太炎所創方案的基礎上制訂了初代「注音字母」。其後,這套方案又經過歷次修改,終於形成了我們現在看到的「注音符號」。}{章太炎模仿日语假名的做法,借用汉字的偏旁等部件创造了若干“纽文”“韵文”,这是注音符号的前身(\cref*{figure:章太炎《駁中國用萬國新語說》「紐文」一頁})。中华民国教育部在章太炎所创方案的基础止制定了初代“注音字母”。其后,这套方案又经过历次修改,终于形成了我们现在看到的“注音符号”。}\par
\content{現今,注音符號在台灣廣泛使用,也是台灣小學生的必修課。而在中國大陸、香港和澳門,注音符號的使用相對較少。}{现今,注音符号在台湾广泛使用,也是台湾小学生的必修课。而在中国大陆、香港和澳门,注音符号的使用相对较少。}\par
\subsubsection{\xpinyin*{\content{漢語拼音}{汉语拼音}}}
\content{如果說注音符號是「假名派」,仿用假名的思路創造聲符;那麼漢語拼音和威妥瑪拼音一樣,都屬於「拉丁派」,使用拉丁字母來創造聲符。}{如果说注音符号是“假名派”,仿用假名的思路创造声符;那么汉语拼音和威妥玛拼音一样,都属于“拉丁派”,使用拉丁字母来创造声符。}\par
\content{漢語拼音系統始於1950年代,是由王力等語言學家整合早期漢字拉丁化基礎與各種拼音方案而開發出的新拼音方案。相比於威妥瑪拼音等舊方案,漢語拼音易於學習和使用(尤其是節省了許多修飾符),無論手寫還是電腦輸入都十分方便。因此對於絕大多數非專業人士來說,漢語拼音都是十分\textbf{簡單}而\textbf{好用}的漢語注音方案。}{汉语拼音系统始于1950年代,是由五力等语言学家整合早期汉字拉丁化基础与各种拼音方案而开发出的新拼音方案。相比于威妥玛拼音等旧方案,汉语拼音易于学习和使用(尤其是省略了许多修饰符),无论手写还是电脑输入都十分方便。因此对于绝大多数非专业人士来说,汉语拼音都是十分\textbf{简单}而\textbf{好用}的汉语注音方案。}\par
\content{現今,漢語拼音方案已經成为漢字轉寫為拉丁字母的規範方式,也是國際普遍承認的漢語拉丁轉寫標準。中國大陸和香港、澳門都將漢語拼音作為小學生的必修課;台灣於2008年起开始採用漢語拼音替代威妥瑪拼音和通用拼音\footnote{通用拼音是台灣的本土拉丁字母拼音方案,於2002年推行,但不久後就被漢語拼音所替代。}。}{现今,汉语拼音方案已经成为汉字转写为拉丁字母的规范方式,也是国际普遍承认的汉语拉丁转写标准。中国大陆和香港、澳门都将汉语拼音作为小学生的必修课;台湾于2008年起开始采用汉语拼音替代威妥玛拼音和通用拼音\footnote{通用拼音是台湾的本土拉丁字母拼音方案,于2002年推行,但不久后就被汉语拼音所替代。}。}\par
\subsection{\content{音節的構成}{音节的构成}}
\content{在我們學習拼音或注音之初,我們就知道:一個字的讀音可以拆成「聲母」和「韻母」兩個部分。當然,有些字或許沒有聲母,比如「安」字只有一個韻母\bopin{ㄢ}{an}。還有些字或許沒有韻母——這種情況在注音系統中保留了,但漢語拼音為它們加上了韻母,我會在後文中介紹。}{在我们学习拼音或注音之初,我们就知道:一个字的读音可以拆成“声母”和“韵母”两个部分。当然,有些字或许没有声母,比如“安”字只有一个韵母\bopin{ㄢ}{an}。还有些字或许没有韵母——这种情况在注音系统中保留了,但汉语拼音为它们加上了韵母,我会在后文中介绍。}\par
\content{為了便於後文對拼音、注音轉換的講解,筆者在此略微介紹一下漢語音節的構成。}{为了便于后文对拼音、注音转换的讲解,笔者在此略微介绍一下汉语音节的构成。}\par
\content{漢語\footnote{這裡所指的「漢語」是現代標準漢語(國語/普通話/標準官話);古漢語及各地方言不在本書的討論範圍內。}是單音節的。對於絕大多數漢字來說\footnote{總是有些特例,比如作為計量單位的「瓩」和「嗧」就是雙音節的漢字,分別讀作\bopin{ㄑㄧㄢ ㄨㄚˇ}{\pinyin{qian1wa3}}和\bopin{ㄐㄧㄚ ㄌㄨㄣˊ}{\pinyin{jia1lun2}}。這些字在港澳台仍然使用;而在中國大陸,它們多半隨著1977年\href{http://www.moe.gov.cn/ewebeditor/uploadfile/2015/01/12/20150112161542547.pdf}{《關於部分計量單位名稱統一用字的通知》}被廢除。},一個字就是一個音節,也就是一個發音單位;而像英語這樣的語言則不同,有些詞是單音節的,有些詞則是多音節的。}{汉语\footnote{这里所指的“汉语”是现代标准汉语(普通话/国语/标准官话);古汉语及各地方言不在本书的讨论范围内。}是单音节的。对于绝大多数汉字来说\footnote{总是有些特例,比如“瓩”和“嗧”就是双音节的汉字,分别读作\bopin{ㄑㄧㄢ ㄨㄚˇ}{\pinyin{qian1wa3}}和\bopin{ㄐㄧㄚ ㄌㄨㄣˊ}{\pinyin{jia1lun2}}。这些字在中国大陆多半随著1977年\href{http://www.moe.gov.cn/ewebeditor/uploadfile/2015/01/12/20150112161542547.pdf}{《关于部分计量单位名称统一用字的通知》}被废除,而在港澳台仍然使用。},一个字就是一个一个音节,也就是一个发音单位;而像英语这样的语言则不同,有些词是单音节的,有些词则是多音节的。}\par
\content{對於漢語來說,一個音節最多只能有一個\textbf{聲母}和一個\textbf{韻母}。而韻母又可以進一步細分成\textbf{韻頭(介音)}和\textbf{韻身}。韻身還可以再細分成韻腹和韻尾,但本書就不深入研究這點了。\cref*{figure:以「窗」為例,分析音節構成}是對「窗」字按此標準劃分出的音節成分。}{对于汉语来说,一个音节最多只能有一个\textbf{声母}和一个\textbf{韵母}。而韵母又可以进一步细分成\textbf{韵头(介音)}和\textbf{韵身}。韵身还可以再细分成韵腹和韵尾,但本书就不深入研究这点了。\cref*{figure:以「窗」為例,分析音節構成}是对“窗”字按此标准划分出的音节成分。}\par
\begin{figure}[H]
    \centering
    \begin{subcaptionblock}{.45\textwidth}
        \centering
        \begin{tikzpicture}[
            level 1/.style={
                level distance=15mm,
                sibling distance=26mm,
            },
            level 2/.style={
                level distance=15mm,
                sibling distance=26mm,
            },
            level 3/.style={
                level distance=15mm,
                sibling distance=26mm,
            },
        ]
            \node{\content{音節}{音节}\boxzhuyin{ㄔㄨㄤ}}
            child{
                node{\content{聲母}{声母}\boxzhuyin{ㄔ}}
            }
            child{
                node{\content{韻母}{韵母}\boxzhuyin{ㄨㄤ}}
                child{
                    node{\content{介音}{介音}\boxzhuyin{ㄨ}}
                }
                child{
                    node{\content{韻身}{韵身}\boxzhuyin{ㄤ}}
                }
            };
        \end{tikzpicture}
        \caption{\content{注音}{注音}}
    \end{subcaptionblock}
    \begin{subcaptionblock}{.45\textwidth}
        \centering
        \begin{tikzpicture}[
            level 1/.style={
                level distance=15mm,
                sibling distance=26mm,
            },
            level 2/.style={
                level distance=15mm,
                sibling distance=26mm,
            },
            level 3/.style={
                level distance=15mm,
                sibling distance=26mm,
            },
        ]
            \node{\content{音節}{音节}\boxpinyin{chuang}}
            child{
                node{\content{聲母}{声母}\boxpinyin{ch}}
            }
            child{
                node{\content{韻母}{韵母}\boxpinyin{uang}}
                child{
                    node{\content{介音}{介音}\boxpinyin{u}}
                }
                child{
                    node{\content{韻身}{韵身}\boxpinyin{ang}}
                }
            };
        \end{tikzpicture}
        \caption{\content{拼音}{拼音}}
    \end{subcaptionblock}
    \caption{\content{以「窗」字為例,分析音節構成}{以“窗”字为例,分析音节构成}}
    \label{figure:以「窗」為例,分析音節構成}
\end{figure}
\content{漢語中的「押韻」,是把韻身相同(或非常相近)的音節視作同一韻部,韻部之內的韻母皆可通押。所以\bopin{ㄓㄤ}{zhang}\bopin{ㄐㄧㄤ}{jiang}\bopin{ㄔㄨㄤ}{chuang}三個音節雖然介音不同,但在新韻\footnote{參見\href{https://zh.wikipedia.org/zh-tw/中華新韻\#\nobreakspace2005年「詩詞學會」版本}{中華新韻\#2005年「詩詞學會」版本 - 維基百科}。}和通韻\footnote{參見\href{http://www.moe.gov.cn/jyb_sjzl/ziliao/A19/202111/W020211118492193544846.pdf}{語言文字規範-中華通韻\ 試行稿}。}兩種韻部體系內,都是可以通押的。}{汉语中的“押韵”,是把韵身相同(或非常相近)的音节视作同一韵部,韵部之内的韵母皆可通押。所以\bopin{ㄓㄤ}{zhang}\bopin{ㄐㄧㄤ}{jiang}\bopin{ㄔㄨㄤ}{chuang}三个音节虽然介音不同,但在新韵\footnote{参见\href{https://zh.wikipedia.org/zh-cn/中華新韻\#\nobreakspace2005年「詩詞學會」版本}{中华新韵\#2005年“诗词学会”版本 - 维基百科}。}和通韵\footnote{参见\href{http://www.moe.gov.cn/jyb_sjzl/ziliao/A19/202111/W020211118492193544846.pdf}{语言文字规范-中华通韵\ 试行稿}。}两种韵部体系内,都是可以通押的。}\par
\content{下面的兩小節將分別介紹漢語拼音與注音符號的相互推導。讀者只需根據自己的需求選讀即可。另請注意,本書不打算做專業的音位分析,所以不會去關注注音符號、漢語拼音二者本身固有的問題\footnote{例如,普通話中\boxpinyin{in}和\boxpinyin{iong}的音值不是 /iən/ 和 /yŋ/,而更接近於 /in/ 和 /yʊŋ/。}。}{下面的两小节将分别介绍汉语拼音与注音符号的相互推导。读者只需根据自己的需求选读即可。另请注ujnf,本书不打算做专业的音位分析,所以不会去关注注音符号、汉语拼音二者本身固有的问题\footnote{例如,普通话中\boxpinyin{in}和\boxpinyin{yong}的音值不是 /iən/ 和 /yŋ/,而更接近于 /in/ 和 /yʊŋ/。}。}
\subsection{\content{由漢語拼音推導注音符號}{由汉语拼音推导注音符号}}
\content{注音符號與漢語拼音之間存在很好的對應關係,所以要把一者「翻譯」成另一者是非常簡單的。對於掌握漢語拼音而初學注音符號的人來說,最大的困難其實是——辨認這些符號並記憶它們的讀音還滿不容易的。}{注音符号与汉语拼音之间存在很好的对应关系,所以要把一者“翻译”成另一者是非常简单的。对于掌握汉语拼音而初学注音符号的人来说,最大的因难其实是——辨认这些符号并记忆它们的读音还是挺不容易的。}\par
\content{筆者認為,了解注音符號的「字源」並與已有的漢字知識建立適當的聯係,是快速掌握注音符號的不二法門。注音符號雖為人造,但也有章法可循,多以古字、異體字或其部件構成,取其聲、韻而得。因此,哪怕只是簡單了解其字源,都可能對學習注音符號大有禆益——至少,有很多符號不需要靠死記硬背來記憶了。}{笔者认为,了解注音符号的“字源”并与已有的汉字知识建立适当的联系,是快速掌握注音符号的不二法门。注音符号虽为人造,但也有音法可循,多以古字、异体字或其部件构成,取其声、韵而得。因此,哪怕只是间单了解其字源,者可能对学习注音符号大有禆益——至少,有很多符号不需要靠死记硬背来记忆了。}\par
\content{\href{https://zh.wikipedia.org/zh-tw/注音符號\#符號字源及發音}{注音符號\#符號字源及發音 - 維基百科}為我們提供了一份詳盡的列表,可供讀者考查字源。}{\href{https://zh.wikipedia.org/zh-cn/注音符號\#符號字源及發音}{注音符号\#符号字源及发音 - 维基百科}为我们提供了一份详尽的列表,可供读者考查字源。}\par
\content{\cref{appendix:注音符號、漢語拼音表}則是筆者根據維基百科及《漢語拼音方案》等資料整理而成的,可以作為注音、拼音轉換的可靠參考。}{\cref{appendix:注音符號、漢語拼音表}则是笔者根据维基百科及《汉语拼音方案》等资料整理而成的,可以作为注音、拼音转换的可靠参考。}\par
\content{但是,筆者並不很喜歡這個表格。它看上去很轉業,也很詳盡,但它的韻母表太過複雜,容易讓讀者望而卻步。筆者更喜歡用「推導」替代「記憶」,即,只需要記住\textbf{必備的基礎}和\textbf{推導的規則},然後把剩下的部分交給推導而非記憶。我們在學習繁簡轉換的過程中將常常與這種方法打交道,比如,若是你知道「馬」的簡化字是「马」,那麼猜想「媽」的簡化字是「妈」就滿合理的。}{但是,笔者并不很喜欢这个表格。它看上去很专业,也很详尽,但它的韵母表太过复杂,容易让读者望而却步。笔者更喜欢用“推导”替代“记忆”,即,只需要记住\textbf{必备的基础}和\textbf{推导的规则},然后把剩下的部分交给推导而非记忆。我们在学习繁简转换的过程中将常常与这种方法打交道,比如,若是你知道“马”的繁体字是“馬”,那么猜想“妈”的繁体字是“媽”就蛮合理的。}\par
\subsubsection{\content{必備的基礎}{必备的基础}}
\content{筆者最初是透過注音鍵盤來接觸注音的。\cref*{figure:大千式注音鍵盤}展示了一種常見的注音鍵盤——大千式注音鍵盤。它把21個聲母、3個介音和13個韻母(其實是韻身)從左到右,依次擺放起來。至於\cref{table:韻母表}中眾多的韻母,它們無非都是由這些介音和韻身組合而成的。}{笔者最初是通过注音键盘来接触注音的。\cref*{figure:大千式注音鍵盤}展示了一种常见的注音键盘——大千式注音键盘。它把21个声母、3个介音和13个韵母(其实是韵身)从左到右,依次摆放起来。至于\cref{table:韻母表}中众多的韵母,它们无非都是由这些介音和韵身组合而成的。}\par
\content{只要了解了這些符號並掌握其拼讀規則,我們就可以用「注音輸入法」來打字了。}{只要了解了这些符号并掌握其拼读规则,我们就可以用“注音输入法”来打字了。}\par
\begin{figure}[H]
    \NewDocumentCommand{\drawkey}{mmmm}{
        \begin{scope}[shift={#1}]
            \draw[fill=white,draw=gray,very thick](0,0)--(1,0)--(1.1,-.1)--(1.1,-1.1)--(.1,-1.1)--(0,-1)--cycle;
            \draw[fill=#2!15,draw=gray,very thick](0,0)rectangle(1,-1);
            \draw (.35,-.42)node[anchor=base west]{#3};
            \draw (0,-.82)node[anchor=base west]{\sffamily\small#4};
        \end{scope}
    }
    \NewDocumentCommand{\drawlegend}{mmm}{
        \begin{scope}[shift={#1}]
            \draw[fill=white,draw=gray,very thick](0,0)--(.5,0)--(.55,-.05)--(.55,-.55)--(.05,-.55)--(0,-.5)--cycle;
            \draw[fill=#2!15,draw=gray,very thick](0,0)rectangle(.5,-.5) node[anchor=base west]{#3};
        \end{scope}
    }
    \centering
    \begin{tikzpicture}
        \drawkey{(0,0)}{red}{ㄅ}{1}
        \drawkey{(1.2,0)}{red}{ㄉ}{2}
        \drawkey{(2.4,0)}{black}{ˇ}{3}
        \drawkey{(3.6,0)}{black}{ˋ}{4}
        \drawkey{(4.8,0)}{red}{ㄓ}{5}
        \drawkey{(6,0)}{black}{ˊ}{6}
        \drawkey{(7.2,0)}{black}{˙}{7}
        \drawkey{(8.4,0)}{blue}{ㄚ}{8}
        \drawkey{(9.6,0)}{blue}{ㄞ}{9}
        \drawkey{(10.8,0)}{blue}{ㄢ}{0}
        \drawkey{(12,0)}{blue}{ㄦ}{-}
        \drawkey{(.4,-1.2)}{red}{ㄆ}{Q}
        \drawkey{(1.6,-1.2)}{red}{ㄊ}{W}
        \drawkey{(2.8,-1.2)}{red}{ㄍ}{E}
        \drawkey{(4,-1.2)}{red}{ㄐ}{R}
        \drawkey{(5.2,-1.2)}{red}{ㄔ}{T}
        \drawkey{(6.4,-1.2)}{red}{ㄗ}{Y}
        \drawkey{(7.6,-1.2)}{green}{ㄧ}{U}
        \drawkey{(8.8,-1.2)}{blue}{ㄛ}{I}
        \drawkey{(10,-1.2)}{blue}{ㄟ}{O}
        \drawkey{(11.2,-1.2)}{blue}{ㄣ}{P}
        \drawkey{(.8,-2.4)}{red}{ㄇ}{A}
        \drawkey{(2,-2.4)}{red}{ㄋ}{S}
        \drawkey{(3.2,-2.4)}{red}{ㄎ}{D}
        \drawkey{(4.4,-2.4)}{red}{ㄑ}{F}
        \drawkey{(5.6,-2.4)}{red}{ㄕ}{G}
        \drawkey{(6.8,-2.4)}{red}{ㄘ}{H}
        \drawkey{(8,-2.4)}{green}{ㄨ}{J}
        \drawkey{(9.2,-2.4)}{blue}{ㄜ}{K}
        \drawkey{(10.4,-2.4)}{blue}{ㄠ}{L}
        \drawkey{(11.6,-2.4)}{blue}{ㄤ}{;}
        \drawkey{(1.2,-3.6)}{red}{ㄈ}{Z}
        \drawkey{(2.4,-3.6)}{red}{ㄌ}{X}
        \drawkey{(3.6,-3.6)}{red}{ㄏ}{C}
        \drawkey{(4.8,-3.6)}{red}{ㄒ}{V}
        \drawkey{(6,-3.6)}{red}{ㄖ}{B}
        \drawkey{(7.2,-3.6)}{red}{ㄙ}{N}
        \drawkey{(8.4,-3.6)}{green}{ㄩ}{M}
        \drawkey{(9.6,-3.6)}{blue}{ㄝ}{,}
        \drawkey{(10.8,-3.6)}{blue}{ㄡ}{.}
        \drawkey{(12,-3.6)}{blue}{ㄥ}{/}
        \drawlegend{(2.5,-5)}{red}{\content{聲母}{声母}}
        \drawlegend{(4.5,-5)}{green}{\content{介音}{介音}}
        \drawlegend{(6.5,-5)}{blue}{\content{韻母(韻身)}{韵母(韻身)}}
        \drawlegend{(10,-5)}{black}{\content{聲調}{声调}}
    \end{tikzpicture}
    \caption{\content{大千式注音輪入法的鍵盤排列}{大千式注音输入法的键盘排列}}
    \label{figure:大千式注音鍵盤}
\end{figure}
\content{注音的聲母與拼音一一對應(唯有拼音中的 i u 是例外),所以我就按照注音鍵盤的順序,列出\cref*{table:注音鍵盤聲母表},以供讀者參考。}{注音的声母与拼音一一对应(唯有拼音中的 i u 是例外),所以我就按照注音键盘的顺序,列出\cref*{table:注音鍵盤聲母表},以供读者参考。}\par
\begin{table}[H]
    \centering
    \caption{\content{注音鍵盤聲母表}{注音键盘声母表}}
    \label{table:注音鍵盤聲母表}
    \begin{NiceTabular}{|*{6}{>{\columncolor{blue!7.5}}c>{\columncolor{red!7.5}}c|}}
        \hline
        ㄅ & b & ㄉ & d & & & & & ㄓ & zh & \\
        ㄆ & p & ㄊ & t & ㄍ & g & ㄐ & j & ㄔ & ch & ㄗ & z \\
        ㄇ & m & ㄋ & n & ㄎ & k & ㄑ & q & ㄕ & sh & ㄘ & c \\
        ㄈ & f & ㄌ & l & ㄏ & h & ㄒ & x & ㄖ & r & ㄙ & s \\\hline
    \end{NiceTabular}
\end{table}
\content{若是縱向閱讀,你或許會發現這些符號的擺放順序很像國小時學習的「ㄅㄆㄇㄈ」「ㄉㄊㄋㄌ」「ㄍㄎㄏ」……的口訣——其實這就是方便使用者記憶鍵位而設定的啦。}{若是纵向阅读,你或许会发现这些符号的摆放顺序很像小学时学习的“b p m f”“d t n l”“g k h”……的口决——其实这就是方便使用者记忆键位而设置的啦。}\par
\content{再來看介音。介音有趣的地方在於,當遇到零聲母\footnote{有些音沒有聲母,比如「安」字唸作\bopin{ㄢ}{an},這就是零聲母。}的情況時,作為「韻頭」的介音可以「退化」成聲母,與「韻身」組合成一個音節。簡單說,\textbf{這時候「介音」的作用是聲母!}}{再来看介音。介音有趣的地方在于,当遇到零声母\footnote{有些音没有声母,比如“安”字念作\bopin{ㄢ}{an},这就是零声母。}的情况时,作为“韵头”的介音可以“退化”成声母,与“韵身”组合成一个音节。简单说,\textbf{这时候“介音”的作用是声母!}}\par
\content{既然介音既可以做韻母,又可以做聲母,那麼注音介音與拼音聲、韻母之間的對照關係也要分成兩種情況來考慮,如\cref*{table:注音鍵盤介音表}所示。}{既然介音既可以做韵母,又可以做声母,那么注音介音与拼音声、韵母的对照关系也要分成两种情况来考虑,如\cref*{table:注音鍵盤介音表}所示。}\par
\begin{table}[H]
    \centering
    \caption{\content{注音鍵盤介音表}{注音键盘介音表}}
    \label{table:注音鍵盤介音表}
    \begin{NiceTabular}{>{\columncolor{blue!7.5}}c*{2}{>{\columncolor{red!7.5}}c}}[hlines]
        \toprule
        \rowcolor{gray!50}
        \content{注音符號}{注音符号} & \content{作韻母的拼音}{作韵母的拼音} & \content{作聲母的拼音}{作声母的拼音} \\\midrule[.6pt]
        ㄧ & i & y \\
        ㄨ & u & w \\
        ㄩ & \"u & \\\bottomrule
    \end{NiceTabular}
\end{table}
\content{注音符號的韻母相比拼音的韻母,看起來要更「少」。比如拼音中的\boxpinyin{en in un \"un}其實都是由同一韻身\bopin{ㄣ}{en}與不同介音組合而成的:}{注音符号的韵母相比拼音的韵母,看起来要更“少”。比如拼音中的\boxpinyin{en in un \"un}其实都是由同一韵身\bopin{ㄣ}{en}与不同介音组合而成的:}
\begin{table}[H]
    \centering
    \begin{NiceTabular}{*{4}{c}}[hlines]
        \toprule
        \rowcolor{gray!50}
        \content{無介音}{无介音} & 介音\bopin{ㄧ}{i} & 介音\bopin{ㄨ}{u} & 介音\bopin{ㄩ}{\"u} \\\midrule[.6pt]
        \bopin{ㄣ}{en} & \bopin{ㄧㄣ}{in} & \bopin{ㄨㄣ}{un} & \bopin{ㄩㄣ}{\"un} \\\bottomrule
    \end{NiceTabular}
\end{table}
\content{也正因如此,這四個韻母在新韻和通韻中都是押韻的。我們從拼音中或許很難看出這一點,但在注音中這是很明顯的!}{也正因如此,这四个韵母在新韵和通韵中都是押韵的。我们从拼音中或许很难看出这一点,但在注音中这是很明显的!}\par
\content{我們在注音鍵盤上看到的韻母(\cref*{table:注音鍵盤韻母表})與拼音中學到的韻母有不少差異,這是初識注音的人需要相當注意的地方。}{我们在注音键盘上看到的韵母(\cref*{table:注音鍵盤韻母表}与拼音中学到的韵母有不少差异,这是初识注音的人需要相当注意的地方。}\par
\begin{table}[H]
    \centering
    \caption{\content{注音鍵盤韻母表}{注音键盘韵母表}}
    \label{table:注音鍵盤韻母表}
    \begin{NiceTabular}{|*{4}{>{\columncolor{blue!7.5}}c>{\columncolor{red!7.5}}c|}}
        \hline
        ㄚ & a & ㄞ & ai & ㄢ & an & ㄦ & er \\
        ㄛ & o & ㄟ & ei & ㄣ & en & & \\
        ㄜ & e & ㄠ & ao & ㄤ & ang & & \\
        ㄝ & \^e & ㄡ & ou & ㄥ & eng & & \\
        \hline
    \end{NiceTabular}
\end{table}
\content{讀者或許會發現,\cref*{figure:大千式注音鍵盤}中還有四個符號:「ˇ\ ˊ\ ˋ\ ˙」。它們也是注音系統的成員——聲調符號。注音和拼音採用相同的聲調符號,所以我們很容易記住每個聲調意味著什麼(\cref*{table:注音聲調表})。}{读者或许会发现,\cref*{figure:大千式注音鍵盤}中还有四个符号:“ˇ\ ˊ\ ˋ\ ˙”。它们也是注音系统的成员——声调符号。注音和拼音采用相同的声调符号,所以我们很容易记信每个声调意味着什么(\cref*{table:注音聲調表})。}\par
\begin{table}[H]
    \centering
    \caption{\content{注音聲調表}{注音声调表}}
    \label{table:注音聲調表}
    \begin{NiceTabular}{*{5}{c}}[hlines]
        \toprule
        \rowcolor{gray!50}
        \content{陰平(一聲)}{阴平(一声)} & \content{陽平(二聲)}{阳平(二声)} & \content{上聲(三聲)}{上声(三声)} & \content{去聲(四聲)}{去声(四声)} & \content{輕聲}{轻声} \\\midrule[.6pt]
        \content{通常不標\footnote{在注音中,不標聲調表示一聲;這和拼音中不標聲調表示輕聲的習慣有所不同。}}{通常不标\footnote{在注音中,不标声调表示一声;这和拼音中不标声调表示轻声的习惯有所不同。}} & ˊ & ˇ & ˋ & ˙ \\\bottomrule
    \end{NiceTabular}
\end{table}
\subsubsection{\content{推導的規則}{推导的规则}}
\content{對於最簡單情況來說,拼音到注音的轉換就是機械性的「聲母、韻母分別轉換」。以「法」字為例,它的拼音是\boxpinyin{\pinyin{fa3}},其聲母是\boxpinyin{f},對應注音中的\boxzhuyin{ㄈ};韻母是\boxpinyin{a},對應注音中的\boxzhuyin{ㄚ};聲調為三聲。那麼注音形式就是\boxzhuyin{ㄈㄚˇ}。}{对于最简单情况来说,拼音到注音的转换就是机械性的“声母、韵母分别转换”。以“法”字为例,它的拼音是\boxpinyin{\pinyin{fa3}},其声母是\boxpinyin{f},对应注音中的\boxzhuyin{ㄈ};韵母是\boxpinyin{a},对应注音中的\boxzhuyin{ㄚ};声调为三声。那么注音形式就是\boxzhuyin{ㄈㄚˇ}。}\par
\content{再以「車」為例,它的一個拼音是\boxpinyin{\pinyin{che1}},其聲母是\boxpinyin{ch},對應注音中的\boxzhuyin{ㄔ};韻母是\boxpinyin{e},對應注音中的\boxzhuyin{ㄜ};聲調為一聲,在注音中一聲通常無需標出。綜上所述,它的注音形式就是\boxzhuyin{ㄔㄜ}。}{再以“车”为例,它的一个拼音是\boxpinyin{\pinyin{che1}},其声母是\boxpinyin{ch},对应注音中的\boxzhuyin{ㄔ};韵母是\boxpinyin{e},对应注音中的\boxzhuyin{ㄜ};声调为一声,在注音中一声通常无需标出。综上所述,它的注音形式就是\boxzhuyin{ㄔㄜ}。}\par
\content{接下來我就著重介紹下「不能以此類推」的幾種情況。}{接下来我就着重介绍下“不能以此类推”的几种情况。}\par
\subsubsubsection{\content{空韻母}{空韵母}}
\content{在學習拼音時,我們一定接觸過一些「整體認讀」的音。以「\xpinyin*{知}」為例,如果要強行把聲母\boxpinyin{zh}和韻母\boxpinyin{i}拼出來,我們卻很難拼出「知」的音節。但是,如果\textbf{只用聲母\boxpinyin{zh}來拼讀,我們卻是可以「拼」出此音的。}}{在学习拼音时,我们一定接触过一些“整体认读”的音。以“\xpinyin*{知}”为例,如果要强行把声母\boxpinyin{zh}和韵母\boxpinyin{i}拼出来,我们却很难拼出“知”的音节。但是,如果\textbf{只用声母\boxpinyin{zh}来拼读,我们却是可以“拼”出此音的。}}\par
\content{對於\boxpinyin{zh ch sh r z c s}七個聲母來說,它們可以獨立成音,而不需要韻母,我們把這種情況稱作空韻母\footnote{實際上它們的韻母是\bopin{ㄭ}{-i},這是一個不同於\bopin{ㄧ}{i}的韻母音。但\bopin{ㄭ}{-i}在注音系統中也一般不會使用,故正文不提。}。所以在為「知」字注音時只需要一個聲母\boxzhuyin{ㄓ}就可以了!}{对于\boxpinyin{zh ch sh r z c s}七个声母来说,它们可以独立成音,而不需要韵母,我们把这种情况称作空韵母\footnote{实际上它们的韵母是\bopin{ㄭ}{-i},这是一个不同于\boxzhuyin{ㄧ}{i}的韵母音。但\bopin{ㄭ}{-i}在注音系统中也一般不会使用,故正文不提。}。所以在为“知”字注音时只需要一个声母\boxzhuyin{ㄓ}就可以了!}\par
\content{同理,「食」「日」「思」的注音分別為\boxzhuyin{ㄕˊ}\boxzhuyin{ㄖˋ}\boxzhuyin{ㄙ}。它們都是空韻母音節。如果非要加上韻母\boxzhuyin{ㄧ},那反而錯了。}{同理,“食”“日”“思”的注音分别为\boxzhuyin{ㄕˊ}\boxzhuyin{ㄖˋ}\boxzhuyin{ㄙ}。它们都是空韵母音节。如果非要加上韵母\boxzhuyin{ㄧ},那反而错了。}\par
\begin{figure}[H]
    \NewDocumentCommand{\drawkey}{mmmm}{
        \begin{scope}[shift={#1}]
            \draw[fill=white,draw=gray,very thick](0,0)--(1,0)--(1.1,-.1)--(1.1,-1.1)--(.1,-1.1)--(0,-1)--cycle;
            \draw[fill=#2!15,draw=gray,very thick](0,0)rectangle(1,-1);
            \draw (.35,-.42)node[anchor=base west]{#3};
            \draw (0,-.82)node[anchor=base west]{\sffamily\small#4};
        \end{scope}
    }
    \centering
    \begin{tikzpicture}
        \drawkey{(0,0)}{black}{ㄅ}{1}
        \drawkey{(1.2,0)}{black}{ㄉ}{2}
        \drawkey{(2.4,0)}{black}{ˇ}{3}
        \drawkey{(3.6,0)}{black}{ˋ}{4}
        \drawkey{(4.8,0)}{red}{ㄓ}{5}
        \drawkey{(6,0)}{black}{ˊ}{6}
        \drawkey{(7.2,0)}{black}{˙}{7}
        \drawkey{(8.4,0)}{black}{ㄚ}{8}
        \drawkey{(9.6,0)}{black}{ㄞ}{9}
        \drawkey{(10.8,0)}{black}{ㄢ}{0}
        \drawkey{(12,0)}{black}{ㄦ}{-}
        \drawkey{(.4,-1.2)}{black}{ㄆ}{Q}
        \drawkey{(1.6,-1.2)}{black}{ㄊ}{W}
        \drawkey{(2.8,-1.2)}{black}{ㄍ}{E}
        \drawkey{(4,-1.2)}{black}{ㄐ}{R}
        \drawkey{(5.2,-1.2)}{red}{ㄔ}{T}
        \drawkey{(6.4,-1.2)}{red}{ㄗ}{Y}
        \drawkey{(7.6,-1.2)}{black}{ㄧ}{U}
        \drawkey{(8.8,-1.2)}{black}{ㄛ}{I}
        \drawkey{(10,-1.2)}{black}{ㄟ}{O}
        \drawkey{(11.2,-1.2)}{black}{ㄣ}{P}
        \drawkey{(.8,-2.4)}{black}{ㄇ}{A}
        \drawkey{(2,-2.4)}{black}{ㄋ}{S}
        \drawkey{(3.2,-2.4)}{black}{ㄎ}{D}
        \drawkey{(4.4,-2.4)}{black}{ㄑ}{F}
        \drawkey{(5.6,-2.4)}{red}{ㄕ}{G}
        \drawkey{(6.8,-2.4)}{red}{ㄘ}{H}
        \drawkey{(8,-2.4)}{black}{ㄨ}{J}
        \drawkey{(9.2,-2.4)}{black}{ㄜ}{K}
        \drawkey{(10.4,-2.4)}{black}{ㄠ}{L}
        \drawkey{(11.6,-2.4)}{black}{ㄤ}{;}
        \drawkey{(1.2,-3.6)}{black}{ㄈ}{Z}
        \drawkey{(2.4,-3.6)}{black}{ㄌ}{X}
        \drawkey{(3.6,-3.6)}{black}{ㄏ}{C}
        \drawkey{(4.8,-3.6)}{black}{ㄒ}{V}
        \drawkey{(6,-3.6)}{red}{ㄖ}{B}
        \drawkey{(7.2,-3.6)}{red}{ㄙ}{N}
        \drawkey{(8.4,-3.6)}{black}{ㄩ}{M}
        \drawkey{(9.6,-3.6)}{black}{ㄝ}{,}
        \drawkey{(10.8,-3.6)}{black}{ㄡ}{.}
        \drawkey{(12,-3.6)}{black}{ㄥ}{/}
    \end{tikzpicture}
	\caption{\content{空韻母對應的注音符號在注音鍵盤中的位置}{空韵母对应的注音符号在注音键盘中的位置}}
	\label{figure:空韻母對應的注音符號在注音鍵盤中的位置}
\end{figure}
\subsubsubsection{\content{韻母推導}{韵母推导}}
\content{我們很容易看出,拼音\boxpinyin{uai}是由介音(韻頭)\boxpinyin{u}與韻身\boxpinyin{ai}結合而成的。我們也不難理解,\boxpinyin{uai}和\boxpinyin{ai}兩個韻母是押韻的,它們都押在\boxpinyin{ai}這個韻身上。}{我们很容易看出,拼音\boxpinyin{uai}是由介音(韵头)\boxpinyin{u}与韵身\boxpinyin{ai}结合而成的。我们也不难理解,\boxpinyin{uai}和\boxpinyin{ai}两个韵母是押韵的,它们都押在\boxpinyin{ai}这个韵身上。}\par
\content{但是,如果我說,\boxpinyin{ei}和\boxpinyin{ui}是押韻的,它們都押在\boxpinyin{ei}這個韻身上,讀者是否會有些遲疑呢?}{但是,如果我说,\boxpinyin{ei}和\boxpinyin{ui}是押韵的,它们都押在\boxpinyin{ei}这个韵身上,读者是否会有些迟疑呢?}\par
\content{其實,\boxpinyin{ui}的完整形式應當是\boxpinyin{uei};只是拼音為了書寫方便之故,將其作了簡化。如果還原成\boxpinyin{uei},那麼我們就很容易理解它的韻身為何——也就很容易推導出它的注音——\boxpinyin{u}對應\boxzhuyin{ㄨ},\boxpinyin{ei}對應\boxzhuyin{ㄟ},所以\boxpinyin{ui}就是\boxzhuyin{ㄨㄟ}啦。}{其实,\boxpinyin{ui}的完整形式应当是\boxpinyin{uei};只是拼音为了书写方便之故,将其作了简化。如果还原成\boxpinyin{uei},那么我们就很容易理解它的韵身为何——也就很容易推导出它的注音——\boxpinyin{u}对应\boxzhuyin{ㄨ},\boxpinyin{ei}对应\boxzhuyin{ㄟ},所以\boxpinyin{ui}就是\boxzhuyin{ㄨㄟ}啦。}\par
\content{類似的原理在\bopin{ㄡ}{ou}\bopin{ㄣ}{en}\bopin{ㄥ}{eng}中也是如此。只要我們「還原」了拼音的完整形式,由拼音推導注音只需做一對一轉換就足矣!(\cref*{table:需要推導的韻母})}{类似的原理在\bopin{ㄡ}{ou}\bopin{ㄣ}{en}\bopin{ㄥ}{eng}中也是如此。只要我们“还原”了拼音的完整形式,由拼音推导注音只需做一对一转换就足矣!(\cref*{table:需要推導的韻母})}\par
\begin{table}[H]
    \centering
    \caption{需要推導的韻母}
    \label{table:需要推導的韻母}
    \begin{NiceTabular}{*{4}{c}}[hlines]
        \toprule
        \rowcolor{gray!50}
        \content{所屬韻身}{所属韵身} & \content{標準形式}{标准形式} & \content{完整形式}{完整形式} & \content{注音符號}{注音符号} \\\midrule[.6pt]
        \bopin{ㄟ}{ei} & \boxpinyin{ui} & \boxpinyin{uei} & \boxzhuyin{ㄨㄟ} \\
        \bopin{ㄡ}{ou} & \boxpinyin{iu} & \boxpinyin{iou} & \boxzhuyin{ㄧㄡ} \\
        \Block{3-1}{\bopin{ㄣ}{en}} & \boxpinyin{in} & \boxpinyin{ien} & \boxzhuyin{ㄧㄣ} \\
        & \boxpinyin{un} & \boxpinyin{uen} & \boxzhuyin{ㄨㄣ} \\ 
        & \boxpinyin{\"un} & \boxpinyin{\"uen} & \boxzhuyin{ㄩㄣ} \\
        \bopin{ㄥ}{eng} & \boxpinyin{ing} & \boxpinyin{ieng} & \boxzhuyin{ㄧㄥ} \\\bottomrule
    \end{NiceTabular}
\end{table}
\begin{figure}[H]
    \NewDocumentCommand{\drawkey}{mmmm}{
        \begin{scope}[shift={#1}]
            \draw[fill=white,draw=gray,very thick](0,0)--(1,0)--(1.1,-.1)--(1.1,-1.1)--(.1,-1.1)--(0,-1)--cycle;
            \draw[fill=#2!15,draw=gray,very thick](0,0)rectangle(1,-1);
            \draw (.35,-.42)node[anchor=base west]{#3};
            \draw (0,-.82)node[anchor=base west]{\sffamily\small#4};
        \end{scope}
    }
    \centering
    \begin{tikzpicture}
        \drawkey{(0,0)}{black}{ㄅ}{1}
        \drawkey{(1.2,0)}{black}{ㄉ}{2}
        \drawkey{(2.4,0)}{black}{ˇ}{3}
        \drawkey{(3.6,0)}{black}{ˋ}{4}
        \drawkey{(4.8,0)}{black}{ㄓ}{5}
        \drawkey{(6,0)}{black}{ˊ}{6}
        \drawkey{(7.2,0)}{black}{˙}{7}
        \drawkey{(8.4,0)}{black}{ㄚ}{8}
        \drawkey{(9.6,0)}{black}{ㄞ}{9}
        \drawkey{(10.8,0)}{black}{ㄢ}{0}
        \drawkey{(12,0)}{black}{ㄦ}{-}
        \drawkey{(.4,-1.2)}{black}{ㄆ}{Q}
        \drawkey{(1.6,-1.2)}{black}{ㄊ}{W}
        \drawkey{(2.8,-1.2)}{black}{ㄍ}{E}
        \drawkey{(4,-1.2)}{black}{ㄐ}{R}
        \drawkey{(5.2,-1.2)}{black}{ㄔ}{T}
        \drawkey{(6.4,-1.2)}{black}{ㄗ}{Y}
        \drawkey{(7.6,-1.2)}{black}{ㄧ}{U}
        \drawkey{(8.8,-1.2)}{black}{ㄛ}{I}
        \drawkey{(10,-1.2)}{blue}{ㄟ}{O}
        \drawkey{(11.2,-1.2)}{blue}{ㄣ}{P}
        \drawkey{(.8,-2.4)}{black}{ㄇ}{A}
        \drawkey{(2,-2.4)}{black}{ㄋ}{S}
        \drawkey{(3.2,-2.4)}{black}{ㄎ}{D}
        \drawkey{(4.4,-2.4)}{black}{ㄑ}{F}
        \drawkey{(5.6,-2.4)}{black}{ㄕ}{G}
        \drawkey{(6.8,-2.4)}{black}{ㄘ}{H}
        \drawkey{(8,-2.4)}{black}{ㄨ}{J}
        \drawkey{(9.2,-2.4)}{black}{ㄜ}{K}
        \drawkey{(10.4,-2.4)}{black}{ㄠ}{L}
        \drawkey{(11.6,-2.4)}{black}{ㄤ}{;}
        \drawkey{(1.2,-3.6)}{black}{ㄈ}{Z}
        \drawkey{(2.4,-3.6)}{black}{ㄌ}{X}
        \drawkey{(3.6,-3.6)}{black}{ㄏ}{C}
        \drawkey{(4.8,-3.6)}{black}{ㄒ}{V}
        \drawkey{(6,-3.6)}{black}{ㄖ}{B}
        \drawkey{(7.2,-3.6)}{black}{ㄙ}{N}
        \drawkey{(8.4,-3.6)}{black}{ㄩ}{M}
        \drawkey{(9.6,-3.6)}{black}{ㄝ}{,}
        \drawkey{(10.8,-3.6)}{blue}{ㄡ}{.}
        \drawkey{(12,-3.6)}{blue}{ㄥ}{/}
    \end{tikzpicture}
    \caption{\content{需要推導的四個韻母在注音鍵盤中的位置}{需要推导的四个韵母在注音键盘中的位置}}
    \label{figure:需要推導的四個韻母在注音鍵盤中的位置}
\end{figure}
\subsubsubsection{\content{零聲母與介音}{零聲母與介音}}
\content{現在我們來思考這四個字的讀音:「\xpinyin*{偶}」「\xpinyin*{有}」「\xpinyin*{狗}」「\xpinyin*{九}」。}{现在我们来思考这四个字的读音:“\xpinyin*{偶}”“\xpinyin*{有}”“\xpinyin*{狗}”“\xpinyin*{九}”。}\par
\content{從拼音中乍一看,好像「偶」「有」「狗」三字都是\boxpinyin{ou}韻母,只有「九」是\boxpinyin{iu}韻母。但是你若仔細唸過,就會發現,其實「有」的韻母和「偶」「狗」不太像,倒是更像「九」!}{从拼音中乍一看,好像“偶”“有”“狗”三字都是\boxpinyin{ou}韵母,只有“九”是\boxpinyin{iu}韵母。但是你若仔细读过,就会发现,其实“有”的韵母和“偶”不太像,倒是更像“九”!}\par
\content{實際上,「有」的韻母是\boxpinyin{iu},而\boxpinyin{iu}的完整形式是\boxpinyin{iou}。這字本來是沒有聲母的(這種現象叫\textbf{零聲母}),是因為介音\boxpinyin{i}變作\boxpinyin{y}充當了聲母,我們才得出「有」字的拼音\boxpinyin{\pinyin{you3}}來。}{实际上,“有”的韵母是\boxpinyin{iu},而\boxpinyin{iu}的完整形式是\boxpinyin{iou}。这字本来是没有声母的(这种现象叫\textbf{零声母}),是因为介音\boxpinyin{i}变作\boxpinyin{y}充当了声母,我们才得出“有”字的拼音\boxpinyin{\pinyin{you3}}来。}\par
\content{而「九」則不同,它本來就有聲母\boxpinyin{j},我們不需要把介音\boxpinyin{i}變成聲母,所以韻母\boxpinyin{iu}的形式才得以保留。}{而“九”则不同,它本来就有声母\boxpinyin{j},我们不需要把介音\boxpinyin{i}变成声母,所以韵母\boxpinyin{iu}的形式才得以保留。}\par
\begin{table}[H]
    \centering
    \caption{\content{「偶」「有」「狗」「九」四字的拼音分析}{“偶”“有”“狗”“九”四字的拼音分析}}
    \label{table:「偶」「有」「狗」「九」四字的拼音分析}
    \begin{NiceTabular}{|>{\columncolor{gray!50}}c|cc|cc|}[corners,hlines]
        \rowcolor{gray!50}
        & \Block{1-2}{\content{韻母\boxpinyin{ou}}{韵母\boxpinyin{ou}}} & & \Block{1-2}{\content{韻母\boxpinyin{iu}}{韵母\boxpinyin{iu}}} & \\
        \content{零聲母}{零声母} & 偶 & \boxpinyin{\pinyin{ou3}} & 有 & \boxpinyin{\pinyin{you3}} \\
        \content{有聲母}{有声母} & 狗 & \boxpinyin{\pinyin{gou3}} & 九 & \boxpinyin{\pinyin{jiu3}} \\
    \end{NiceTabular}
\end{table}
\content{而注音對零聲母現象的處理方式就比較直觀:}{而注音对零声母现象的处理方式就比较直观:}
\begin{table}[H]
    \centering
    \caption{\content{「偶」「有」「狗」「九」四字的注音分析}{“偶”“有”“狗”“九”四字的注音分析}}
    \label{table:「偶」「有」「狗」「九」四字的注音分析}
    \begin{NiceTabular}{|>{\columncolor{gray!50}}c|cc|cc|}[corners,hlines]
        \rowcolor{gray!50}
        & \Block{1-2}{\content{韻母\boxzhuyin{ㄡ}}{韵母\boxzhuyin{ㄡ}}} & & \Block{1-2}{\content{韻母\boxzhuyin{ㄧㄡ}}{韵母\boxzhuyin{ㄧㄡ}}} & \\
        \content{零聲母}{零声母} & 偶 & \boxzhuyin{ㄡˇ} & 有 & \boxzhuyin{ㄧㄡˇ} \\
        \content{有聲母}{有声母} & 狗 & \boxzhuyin{ㄍㄡˇ} & 九 & \boxzhuyin{ㄐㄧㄡˇ} \\
    \end{NiceTabular}
\end{table}
\content{當介音\boxpinyin{u}作聲母時,它將變成\boxpinyin{w}。所以\boxpinyin{wei}其實是零聲母的\boxpinyin{uei}。因此,「威」字的注音是\boxzhuyin{ㄨㄟ},它和有聲母的「規\boxzhuyin{ㄍㄨㄟ}」是相同韻母的。}{当介音\boxpinyin{u}作声母时,它将变成\boxpinyin{w}。所以\boxpinyin{wei}其实是零声母的\boxpinyin{uei}。因此,“威”字的注音是\boxzhuyin{ㄨㄟ},它和有声母的“规\boxzhuyin{ㄍㄨㄟ}”是相同韵母的。}\par
\content{當介音\boxpinyin{\"u}\footnote{提醒讀者,當\boxpinyin{\"u}以\boxpinyin{j q x y}為前綴時,\boxpinyin{\"u}的寫法將簡化為\boxpinyin{u}。所以拼音中出現的\boxpinyin{u}究竟是\boxpinyin{u}還是\boxpinyin{\"u},需要讀者自行甄別。}作聲母時,情況有些不同。在拼音中,我們對它的處理方式是為它加上一個偽聲母\boxpinyin{y}\footnote{這或許是因為,撮口呼的口型本身就包含了齊齒呼的口型,所以\boxpinyin{y}音將被\boxpinyin{\"u}覆蓋,不起作用。}。所以\boxpinyin{yuan}其實是\boxpinyin{\"uan},那麼「元」的注音為\boxzhuyin{ㄩㄢˊ}就不難理解了!}{当介音\boxpinyin{\"u}\footnote{提醒读者,当\boxpinyin{\"u}以\boxpinyin{j q x y}为前缀时,\boxpinyin{\"u}的写法将简化为\boxpinyin{u}。所以拼音中出现的\boxpinyin{u}究竟是\boxpinyin{u}还是\boxpinyin{\"u},需要读者自行甄别。}作声母时,情况有些不同。在拼音中,我们对它的处理方式是为它加上一个伪声母\boxpinyin{y}\footnote{这或许是因为,撮口呼的口型本身就包含了齐齿呼的口型,所以\boxpinyin{y}音将被\boxpinyin{\"u}覆盖,不起作用。}。}\par
\subsubsubsection{\content{\bopin{ㄜ}{e}與\bopin{ㄝ}{\^e}}{\bopin{ㄜ}{e}与\bopin{ㄝ}{\^e}}}
\content{拼音中\boxpinyin{ie}\boxpinyin{\"ue}兩韻母的韻身其實不是\bopin{ㄜ}{e},而是\bopin{ㄝ}{\^e},只是出於方便書寫的目的才寫作\boxpinyin{e}的。因而\boxpinyin{ie}的注音是\boxzhuyin{ㄧㄝ}而不是\boxzhuyin{ㄧㄜ};\boxpinyin{\"ue}的注音是\boxzhuyin{ㄩㄝ}而不是\boxzhuyin{ㄩㄜ}。}{拼音中\boxpinyin{ie}\boxpinyin{\"ue}两韵母是韵身其实不是\bopin{ㄜ}{e},而是\bopin{ㄝ}{\^e},只是出于方便书写的目的才写作\boxpinyin{e}的。因而\boxpinyin{ie}的注音是\boxzhuyin{ㄧㄝ}而不是\boxzhuyin{ㄧㄜ};\boxpinyin{\"ue}的注音是\boxzhuyin{ㄩㄝ}而不是\boxzhuyin{ㄩㄜ}。}\par
\content{至於\bopin{ㄜ}{e},其實它原本與\bopin{ㄛ}{o}發音相似,被視為同音位,後來才分化出來。所以在新韻中,\bopin{ㄜ}{e}與\bopin{ㄛ}{o}可以通押而與\bopin{ㄝ}{\^e}不可通押。}{至于\bopin{ㄜ}{e},其实它原本与\bopin{ㄛ}{o}相似,被视为同音位,后来才分化出来。所以在新韵中,\bopin{ㄜ}{e}与\bopin{ㄛ}{o}可以通押而与\bopin{ㄝ}{\^e}不可通押。}\par
\subsubsubsection{\content{特殊的\bopin{ㄨㄥ}{ong}}{特殊的\bopin{ㄨㄥ}{ong}}}
\content{對於前文所述的韻母推導規則來說,\bopin{ㄨㄥ}{ong}算是一個特例。}{对于前文所述的韵母推导规则来说,\bopin{ㄨㄥ}{ong}算是一个特例。}\par
\content{在通韻中,拼音的\boxpinyin{eng}和\boxpinyin{ong}是兩個韻部,不能通押。從發音本身來看,\boxzhuyin{ㄨ}和\boxzhuyin{ㄥ}拼起來的\boxpinyin{ueng}也與\boxpinyin{ong}音有明顯差異。但是,注音系統並沒有區分這兩個音,只是規定它們在不同場合下讀法不同:零聲母時讀成\boxpinyin{weng},有聲母時讀成\boxpinyin{ong}。}{在通韵中,拼间的\boxpinyin{eng}和\boxpinyin{ong}是两个韵部,不能通押。从发音本身来看,\boxzhuyin{ㄨ}和\boxzhuyin{ㄥ}拼起来的\boxpinyin{ueng}也与\boxpinyin{ong}音有明显差异。但是,注音系统并没有区分这两个音,只是规定它们在不同场合下读法不同:零声母时读成\boxpinyin{weng},有声母时读成\boxpinyin{ong}。}\par
\content{另外,\boxpinyin{iong}的注音寫法也不是單純加個\boxpinyin{i},寫成\boxzhuyin{ㄧㄨㄥ};而是把\boxzhuyin{ㄨ}改成了\boxzhuyin{ㄩ},寫成\boxzhuyin{ㄩㄥ}。所以「用」字的注音是\boxzhuyin{ㄩㄥˋ};「兄」字的注音是\boxzhuyin{ㄒㄩㄥ}。}{另外,\boxpinyin{iong}的注音写法也不是单纯加个\boxpinyin{i},写成\boxzhuyin{ㄧㄨㄥ};而是把\boxzhuyin{ㄨ}改成了\boxzhuyin{ㄩ},写成\boxzhuyin{ㄩㄥ}。所以“用”字的注音是\boxzhuyin{ㄩㄥˋ};“兄”字的注音是\boxzhuyin{ㄒㄩㄥ}。}\par
\subsection{\content{由注音符號推導漢語拼音}{由注音符号推导汉语拼音}}
\content{漢語拼音與注音符號之間存在很好的對應關係,所以要把一者「翻譯」成另一者是非常簡單的。對於有一定英文基礎的人來說,記憶拼音也是相當容易的——因為我們根據英文字母的讀法,就可以把一個拼音字母的讀音猜得八九不離十了。}{汉语拼音与注音符号之间存在很好的对应关系,所以要把一者“翻译”成另一者是非常简单的。对于有一定英文基础的人来说,记忆拼音也是相当容易的——因为我们根据英文字母的读法,就可以把一个拼音字母的读音猜得八九不离十了。}\par
\content{但是讀者也要當心聰明反被聰明誤。就拿拼音\boxpinyin{q}來說,實際上英文中的q常發類似於\bopin{ㄎ}{k}的音(如unique, Iraq等詞)。那麼我們就有可能受此誤導,把\boxpinyin{q}對應到\boxzhuyin{ㄎ}上去了。但它其實應該是\boxzhuyin{ㄑ}才對。}{但是读者也要当心聪明反被聪明误。就拿拼音\boxpinyin{q}来说,实际上英文中的q常发类似于\bopin{ㄎ}{k}的音(如unique, Iraq等词)。那么我们就有可能受此误导,把\boxpinyin{q}对应到\boxzhuyin{ㄎ}上去了。但它其实应该是\boxzhuyin{ㄑ}才对。}\par
\content{再如,找們還可能受英文及威妥瑪拼音的影響,把\boxzhuyin{ㄑ}拼成\boxpinyin{chi}。類似的這些錯誤都是「籍由英文記憶拼音」時容易犯的,讀者需要留心。}{再如,我们还可能受英文及威妥玛拼音的影响,把\boxzhuyin{ㄑ}拼成\boxpinyin{chi}。类似的这些错误都是“借由英文记忆拼音”时容易犯的,读者需要留心。}\par
\content{\cref{appendix:注音符號、漢語拼音表}是筆者根據維基百科及《漢語拼音方案》等資料整理而成的,可以作為注音、拼音轉換的可靠參考。}{\cref{appendix:注音符號、漢語拼音表}是笔者根据维基百科及《汉语拼音方案》等资料整理而成的,可以作为注音、拼音转换的可靠参考。}\par
\content{但是,筆者並不很喜歡這個表格。它看上去很轉業,也很詳盡,但它的韻母表太過複雜,容易讓讀者望而卻步。筆者更喜歡用「推導」替代「記憶」,即,只需要記住\textbf{必備的基礎}和\textbf{推導的規則},然後把剩下的部分交給推導而非記憶。我們在學習繁簡轉換的過程中將常常與這種方法打交道,比如,若是你知道「馬」的簡化字是「马」,那麼猜想「媽」的簡化字是「妈」就滿合理的。}{但是,笔者并不很喜欢这个表格。它看上去很专业,也很详尽,但它的韵母表太过复杂,容易让读者望而却步。笔者更喜欢用“推导”替代“记忆”,即,只需要记住\textbf{必备的基础}和\textbf{推导的规则},然后把剩下的部分交给推导而非记忆。我们在学习繁简转换的过程中将常常与这种方法打交道,比如,若是你知道“马”的繁体字是“馬”,那么猜想“妈”的繁体字是“媽”就蛮合理的。}\par
\subsubsection{\content{必備的基礎}{必备的基础}}
\content{現行的注音體系\footnote{既不包含\boxzhuyin{ㄪㄫㄬ}等老國音聲母,也不使用韻符\boxzhuyin{ㄭ}。}包含21個聲母、3個介音和13個韻母(其實是韻身)。若是要做推導,我們至少需要先記住這37個符號對應的拼音,然後才談得上推導。}{现行的注音体系\footnote{既不包含\boxzhuyin{ㄪㄫㄬ}等老国音声母,也不使用韵符\boxzhuyin{ㄭ}。}包含21个声母、3个介音和13个韵母(其实是韵身)。若是要做推导,我们至少需要先记信这37个符号对应的拼音,然后才谈得上推导。}\par
\content{21聲母和13韻母都與拼音一一對應:}{21声母和13韵母都与拼音一一对应:}
\begin{table}[H]
    \centering
    \caption{\content{注音聲、韻母與拼音對照表}{注音声、韵母与拼音对照表}}
    \label{table:注音聲、韻母與拼音對照表}
    \begin{subcaptionblock}{.6\textwidth}
        \centering
        \begin{NiceTabular}{|*{4}{>{\columncolor{blue!7.5}}c>{\columncolor{red!7.5}}c|}}
            \hline
            ㄦ & er & ㄢ & an & ㄞ & ai & ㄚ & a \\
            & & ㄣ & en & ㄟ & ei & ㄛ & o \\
            & & ㄤ & ang & ㄠ & ao & ㄜ & e \\
            & & ㄥ & eng & ㄡ & ou & ㄝ & \^e \\\hline
        \end{NiceTabular}
        \caption{\content{韻母部分}{韵母部分}}
    \end{subcaptionblock}
    \begin{subcaptionblock}{.8\textwidth}
        \centering
        \begin{NiceTabular}{|*{6}{>{\columncolor{blue!7.5}}c>{\columncolor{red!7.5}}c|}}
            \hline
            ㄗ & z & ㄓ & zh & ㄐ & j & ㄍ & g & ㄉ & d & ㄅ & b \\
            ㄘ & c & ㄔ & ch & ㄑ & q & ㄎ & k & ㄊ & t & ㄆ & p \\
            ㄙ & s & ㄕ & sh & ㄒ & x & ㄏ & h & ㄋ & n & ㄇ & m \\
            & & ㄖ & r & & & & & ㄌ & l & ㄈ & f \\
            \hline
        \end{NiceTabular}
        \caption{\content{聲母部分}{声母部分}}
    \end{subcaptionblock}
\end{table}
\content{唯有介音的規則比較複雜。在使用介音時,我們一般是把它當作韻母的一部分(也就是韻頭)來用;但當整個音節沒有聲母時,介音可以「退化」為聲母,與韻身部分組成一個音節。在注音系統中,我們不需要關注這些細節;但在拼音系統中,介音作聲母和作韻母時卻有不同的寫法。}{唯有介音的规则比较复杂。在使用介音时,我们一般是把它当作韵母的一部分(也就是韵头)来用;但当整个音节没有声母时,介音可以“退化”为声母,与韵身部分组成一个音节。在注音系统中,我们不需要关注这些细节;但在拼音系统中,介音作声母和作韵母时却有不同的写法。}\par
\begin{table}[H]
    \centering
    \caption{\content{注音介母與拼音對照表}{注音介母与拼音对照表}}
    \label{table:注音介母與拼音對照表}
    \begin{NiceTabular}{>{\columncolor{blue!7.5}}c*{2}{>{\columncolor{red!7.5}}c}}[hlines]
        \toprule
        \rowcolor{gray!50}
        \content{注音符號}{注音符号} & \content{作韻母的拼音}{作韵母的拼音} & \content{作聲母的拼音}{作声母的拼音} \\\midrule[.6pt]
        ㄧ & i & y \\
        ㄨ & u & w \\
        ㄩ & \"u & \\\bottomrule
    \end{NiceTabular}
\end{table}
\content{最後是聲調的問題。漢語拼音中的聲調習慣與注音和威妥瑪拼音都不同,不是把音調記號放在字母的右上方,而是放在某一個字母的正上方,像\bopin{ㄔㄨㄤ}{\pinyin{chuang1}}這樣。而值得探究的問題是,聲調符號到底要標在哪個字母的上方呢?我們稍後再來談。}{最后是声调的问题。汉语拼音中的声调习惯与注音和威妥玛拼音都不同,不是把音调记号放在字母的右上方,而是放在某一个字母的正上方,像\bopin{ㄔㄨㄤ}{\pinyin{chuang1}}这样。而值得探究的问题是,声调符号到底要标在哪个字母的上方呢?我们稍后再来谈。}\par
\content{另外,漢語拼音中的一聲(陰平)是需要用一橫來標出的,這點也與注音的習慣有所不同。在漢語拼音中,輕聲則是不標聲調的。}{另外,汉语拼音中的一声(阴平)是需要用一横来标出的,这点也与注音的习惯有所不同。在汉语拼音中,轻声则是不标声调的。}\par
\subsubsection{\content{推導的規則}{推导的规则}}
\content{對於最簡單情況來說,注音到拼音的轉換就是機械性的「聲母、韻母分別轉換」。以「法」字為例,它的注音是\boxzhuyin{ㄈㄚˇ},其聲母是\boxzhuyin{ㄈ},對應拼音中的\boxpinyin{f};韻母是\boxzhuyin{ㄚ},對應拼音中的\boxpinyin{a};聲調為三聲,聲調需要標在韻母的頭頂。所以拼音形式就是\boxpinyin{\pinyin{fa3}}。}{对于最简单情况来说,注音到拼音的转换就是机械性的“声母、韵母分别转换”。以“法”字为例,它的注音是\boxzhuyin{ㄈㄚˇ},其声母是\boxzhuyin{ㄈ},对应拼音中的\boxpinyin{f};韵母是\boxzhuyin{ㄚ},对应注音中的\boxpinyin{a};声调为三声,声调需要标在韵母的头顶。所以拼音形式就是\boxpinyin{\pinyin{fa3}}。}\par
\content{再以「車」字為例,它的一個注音是\boxzhuyin{ㄔㄜ},其聲母是\boxzhuyin{ㄔ},對應拼音中的\boxpinyin{ch};韻母是\boxzhuyin{ㄜ},對應拼音中的\boxpinyin{e};聲調為一聲,一聲在拼音中是需要標出的。綜上所述,它的拼音形式就是\boxpinyin{\pinyin{che1}}。}{再以“车”字为例,它的一个注音是\boxzhuyin{ㄔㄜ},其声母是\boxzhuyin{ㄔ},对应拼音中的\boxpinyin{ch};韵母是\boxzhuyin{ㄜ},对应拼音中的\boxpinyin{e};声调为一声,一声在拼音中是需要标出的。综上所述,它的拼音形式就是\boxpinyin{\pinyin{che1}}。}\par
\content{接下來我就著重介紹下「不能以此類推」的幾種情況。}{接下来我就着重介绍下“不能以此类推”的几种情况。}\par
\subsubsubsection{\content{空韻母補全}{空韵母补全}}
\content{在注音系統中,有些音節是沒有聲母的。就以「知\boxzhuyin{ㄓ}」來說,它就是一個只需聲母,無需韻母就可以拼讀出來的字。我們把這種情況稱作空韻母\footnote{實際上它的韻母是\bopin{ㄭ}{-i},這是一個不同於\bopin{ㄧ}{i}的韻母音。但\bopin{ㄭ}{-i}在注音系統中也一般不會使用,故正文不提。}。}{在注音系統中,有些音节是没有声母的。就以”知\boxzhuyin{ㄓ}来说,它就是一个只需声母,无需韵母就可以拼读出来的字。我们把这种情况称作空韵母\footnote{实际上它们的韵母是\bopin{ㄭ}{-i},这是一个不同于\bopin{ㄧ}{i}的韵母音。但\bopin{ㄭ}{-i}在注音系统中也一般不会使用,故正文不提。}。}\par
\content{拼音系統則為空韻母音節補全了韻母\footnote{至於個別的「嗯\bopin{ㄋˋ}{\`ng}」這樣的字,就不在討論範圍內了。}。對於\boxzhuyin{ㄓ ㄔ ㄕ ㄖ ㄗ ㄘ ㄙ}七個聲母而言,我們在轉換成拼音之後要為它加上韻母\boxpinyin{i}\footnote{其實應該加\boxpinyin{-i}。但正如前文所說,漢語拼音是「簡單」的系统,我們只需要寫成\boxpinyin{i}就夠了。}。所以「知」「食」「日」「思」的拼音分別為\boxpinyin{\pinyin{zhi1}}\boxpinyin{\pinyin{shi2}}\boxpinyin{\pinyin{ri4}}\boxpinyin{\pinyin{si1}}。}{拼音系统则为空韵母音节补全了韵母\footnote{至于个别的「嗯\bopin{ㄋˋ}{\`ng}」这样的字,就不在讨论范围内了。}。对于\boxzhuyin{ㄓ ㄔ ㄕ ㄖ ㄗ ㄘ ㄙ}七个声母而言,我们在转换成拼音之后要为它加上韵母\boxpinyin{i}\footnote{其实应该加\boxpinyin{-i}。但正如前文所说,汉语拼音是“简单”的系统,我们人需要写成\boxpinyin{i}就够了。}。所以“知”“食”“日”“思”的拼音分别为\boxpinyin{\pinyin{zhi1}}\boxpinyin{\pinyin{shi2}}\boxpinyin{\pinyin{ri4}}\boxpinyin{\pinyin{si1}}。}\par
\subsubsubsection{\content{韻母簡化}{韻母簡化}}
\content{然後讓我們來聚焦介音。在沒有介音時,注音到拼音的轉換就像前面介紹的那樣,簡單地做一對一轉換足矣。但當存在介音時,情況變得複雜許多。以「村\boxzhuyin{ㄘㄨㄣ}」字為例,如果要一對一地轉換,它的拼音應該是\boxpinyin{cuen}。不過,為了簡單起見,漢語拼音把韻母\boxpinyin{uen}簡化成了\boxpinyin{un}。雖然\boxpinyin{cuen}的寫法並不算錯誤,但你若是用這種寫法去與一般的拼音使用者交流,他們大概是不明白你要表達的意思的。}{然后让我们来聚焦介音。在没有介音时,注音到拼音的转换就像前面介绍的那样,简单地做一对一转换足矣。但当存在介音时,情况变得复杂许多。以“村\boxzhuyin{ㄘㄨㄣ}”字为例,如果要一对一地转换,它的拼音应该是\boxpinyin{cuen}。不过,为了简单起见,汉语拼音把韵母\boxpinyin{uen}简化成了\boxpinyin{un}。虽然\boxpinyin{cuen}的写法并不算错误,但你若是用这种写法去与一般的拼音使用者交流,他们大概是不明白你要表达的意思的。}\par
\content{在漢語拼音中,有四個這樣的韻身。它們在遇到介音時,要做一定的簡化,詳見\cref*{table:需要簡化的韻母}。}{在汉语拼音中,有四个这样的韵身。它们在遇到介音时,要做一定的简化。详见\cref*{table:需要簡化的韻母}。}\par
\begin{table}[H]
    \centering
    \caption{\content{需要簡化的韻母}{需要简化的韵母}}
    \label{table:需要簡化的韻母}
    \begin{NiceTabular}{*{4}{c}}[hlines]
        \toprule
        \rowcolor{gray!50}
        \content{所屬韻身}{所属韵身} & \content{注音符號}{注音符號} & \content{拼音轉寫}{拼音转写} & \content{標準拼音}{标准拼音} \\\midrule[.6pt]
        \bopin{ㄟ}{ei} & \boxzhuyin{ㄨㄟ} & \boxpinyin{uei} & \boxpinyin{ui} \\
        \bopin{ㄡ}{ou} & \boxzhuyin{ㄨㄡ} & \boxpinyin{iou} & \boxpinyin{iu} \\
        \Block{3-1}{\bopin{ㄣ}{en}} & \boxzhuyin{ㄧㄣ} & \boxpinyin{ien} & \boxpinyin{in} \\
        & \boxzhuyin{ㄨㄣ} & \boxpinyin{uen} & \boxpinyin{un} \\
        & \boxzhuyin{ㄩㄣ} & \boxpinyin{\"uen} & \boxpinyin{\"un} \\
        \bopin{ㄥ}{eng} & \boxzhuyin{ㄧㄥ} & \boxpinyin{ieng} & \boxpinyin{ing} \\\bottomrule
    \end{NiceTabular}
\end{table}
\subsubsubsection{\content{零聲母與介音}{零声母与介音}}
\content{現在我們來分析四個字:「偶」「有」「狗」「九」。直觀起見,請讓我用\cref*{table:「偶」「有」「狗」「九」四字的注音分析*}說明之。}{现在我们来分析四个字:“偶”“有”“狗”“九”。直观起见,请让我用\cref*{table:「偶」「有」「狗」「九」四字的注音分析*}说明之。}\par
\begin{table}[H]
    \centering
    \caption{\content{「偶」「有」「狗」「九」四字的注音分析}{“偶”“有”“狗”“九”四字的注音分析}}
    \label{table:「偶」「有」「狗」「九」四字的注音分析*}
    \begin{NiceTabular}{|>{\columncolor{gray!50}}c|cc|cc|}[corners,hlines]
        \rowcolor{gray!50}
        & \Block{1-2}{\content{韻母\boxzhuyin{ㄡ}}{韵母\boxzhuyin{ㄡ}}} & & \Block{1-2}{\content{韻母\boxzhuyin{ㄧㄡ}}{韵母\boxzhuyin{ㄧㄡ}}} & \\
        \content{零聲母}{零声母} & 偶 & \boxzhuyin{ㄡˇ} & 有 & \boxzhuyin{ㄧㄡˇ} \\
        \content{有聲母}{有声母} & 狗 & \boxzhuyin{ㄍㄡˇ} & 九 & \boxzhuyin{ㄐㄧㄡˇ} \\
    \end{NiceTabular}
\end{table}
\content{韻母\boxzhuyin{ㄡ}在\cref*{table:需要簡化的韻母}中「需要簡化的韻母」之列。一般來說,當它單獨存在作為韻母時,對應拼音\boxpinyin{ou};但若存在介音\boxpinyin{i},就需要簡寫為\boxpinyin{iu}了。}{韵母\boxzhuyin{ㄡ}在\cref*{table:「偶」「有」「狗」「九」四字的注音分析*}中“需要简化的韵母”之列。一般来说,当它单独存在作为韵母时,对应拼音\boxpinyin{ou};但若存在介音\boxpinyin{i},就需要简写为\boxpinyin{iu}了。}\par
\content{但當整個音節沒有聲母(這種現象叫零聲母)時,情況又有所不同。還記得我們之前的說法嗎?在沒有聲母時,介音可以退化為聲母。於是參考\cref*{table:注音介母與拼音對照表}可以知道,\boxzhuyin{ㄧㄡ}中的介音\boxzhuyin{ㄧ}應該轉寫為\boxpinyin{y}而不是\boxpinyin{i}!}{但当整个音节没有声母(这种现象叫零声母)时,情况又有所不同。还记得我们之前的说法吗?在没有声母时,介音可以退化为声母。于是参考\cref*{table:注音介母與拼音對照表}可以知道,\boxzhuyin{ㄧㄡ}中的介音\boxzhuyin{ㄧ}应该转写为\boxpinyin{y}而不是\boxpinyin{i}!}\par
\content{既然\boxzhuyin{ㄧ}已經充當了聲母,那麼它就沒有理由參與「韻母簡化」了。所以「有」字的拼音就是\boxpinyin{\pinyin{you3}}。}{既然\boxzhuyin{ㄧ}已经充当了声母,那么它就没有理由参考“韵母简化”了。所以“有”字的拼音就是\boxpinyin{\pinyin{you3}}。}\par
\content{再來看「九」字。它原本就有聲母\bopin{ㄐ}{j},不需要介音\boxzhuyin{ㄧ}來充當聲母,因此韻母\boxzhuyin{ㄧㄡ}可以正常簡化為\boxpinyin{iu}。所以“九”字的拼音就是\boxpinyin{\pinyin{jiu3}}。}{再来看“九”字。它原本就有声母\bopin{ㄐ}{j},不需要介音\boxzhuyin{ㄧ}来充当声母,因此韵母\boxzhuyin{ㄧㄡ}可以正常简化为\boxpinyin{iu}。所以“九”字的拼音就是\boxpinyin{\pinyin{jiu3}}。}\par
\begin{table}[H]
    \centering
    \caption{\content{「偶」「有」「狗」「九」四字的拼音分析}{“偶”“有”“狗”“九”四字的拼音分析}}
    \label{table:「偶」「有」「狗」「九」四字的拼音分析*}
    \begin{NiceTabular}{|>{\columncolor{gray!50}}c|cc|cc|}[corners,hlines]
        \rowcolor{gray!50}
        & \Block{1-2}{\content{韻母\boxpinyin{ou}}{韵母\boxpinyin{ou}}} & & \Block{1-2}{\content{韻母\boxpinyin{iu}}{韵母\boxpinyin{iu}}} & \\
        \content{零聲母}{零声母} & 偶 & \boxpinyin{\pinyin{ou3}} & 有 & \boxpinyin{\pinyin{you3}} \\
        \content{有聲母}{有声母} & 狗 & \boxpinyin{\pinyin{gou3}} & 九 & \boxpinyin{\pinyin{jiu3}} \\
    \end{NiceTabular}
\end{table}
\content{當介音\boxzhuyin{ㄨ}作聲母時,它將變成\boxpinyin{w}。所以\boxzhuyin{ㄨㄟ}這個音節的對應拼音是\boxpinyin{wei}。}{当介音\boxzhuyin{ㄨ}作声母时,它将变成\boxpinyin{w}。所以\boxzhuyin{ㄨㄟ}这个音节的对应拼音是\boxpinyin{wei}。}\par
\content{至於介音\bopin{ㄩ}{\"u},它比較特殊。當\boxzhuyin{ㄩ}出現在零聲母音節中時,我們要做的不是把\boxpinyin{\"u}替換成什麼,而是為它加上一個偽聲母\boxpinyin{y}。以「元\boxzhuyin{ㄩㄢˊ}」字為例。我們先一一對應地把它轉寫成拼音,即\boxpinyin{\"u\'an}。因為它沒有聲母,所以我們需要按照零聲母的處理方式,為\boxpinyin{\"u}加上一個偽聲母\boxpinyin{y},從而寫成\boxpinyin{y\"u\'an}。其實它的更標準寫法是\boxpinyin{\pinyin{yuan2}},這點我們稍後再談。}{至于介音\bopin{ㄩ}{\"u},它比较特殊。当\boxzhuyin{ㄩ}出现在零声母音节中时,我们要做的不是把\boxpinyin{\"u}替换成什么,而是为它加上一个伪声母\boxpinyin{y}。以“元\boxzhuyin{ㄩㄢˊ}为例。我们先一一对应地把它转写成拼音,即\boxpinyin{\"u\'an}。因为它没有声母,所以我们需要按照零声母的处理方式,为\boxpinyin{\"u}加上一个伪声母\boxpinyin{y},从而写成\boxpinyin{y\"u\'an}。其实它的更标准写法是\boxpinyin{\pinyin{yuan2}},这点我们稍后再谈。}\par
\content{還有一類零聲母的情形是既沒有聲母也沒有韻身,由介音單獨構成一個音節。這個現象對應的音節只有三個:\boxzhuyin{ㄧ}\boxzhuyin{ㄨ}\boxzhuyin{ㄩ}。當轉寫為拼音時,我們需要為它們添加偽聲母\boxpinyin{y}或\boxpinyin{w}。比如說,「衣」「物」「魚」應該拼作\bopin{ㄧ}{\pinyin{yi1}}\bopin{ㄨˋ}{\pinyin{wu4}}\bopin{ㄩˊ}{\pinyin{yu2}}。}{还有一类零声母的情形是既没有声母也没有韵母,由介音单独构成一个音节。这个现象对应的音节只有三个:\boxzhuyin{ㄧ}\boxzhuyin{ㄨ}\boxzhuyin{ㄩ}。当转写为拼音时,我们需要为它们添加伪声母\boxpinyin{y}或\boxpinyin{w}。比如说,“衣”“物”“鱼”应该拼作\bopin{ㄧ}{\pinyin{yi1}}\bopin{ㄨˋ}{\pinyin{wu4}}\bopin{ㄩˊ}{\pinyin{yu2}}。}\par
\subsubsubsection{\content{\boxpinyin{\"u}與\boxpinyin{u};\boxpinyin{\^e}與\boxpinyin{e}}{\boxpinyin{\"u}与\boxpinyin{u};\boxpinyin{\^e}与\boxpinyin{e}}}
\content{我們反覆提到,漢語拼音的特點就是「簡潔」。對於一些較麻煩的細節,漢語拼音做了不少的簡化。雖然有時這樣的簡化可能會誤導我們對音節的分析,但是對於普羅大眾的日常使用來說,做到這個程度就足夠了。}{我们反复提到,汉语拼音的特点就是“简洁”。对于一些较麻烦的细节,汉语拼音做了不少的简化。虽然有时这样的简化可能会误导我们对音节的分析,但是对于普罗大众的日常使用来说,做到这个程序就足够了。}\par
\content{「簡潔」的要義之一就是,在不致引起歧義的情況下,略去一些麻煩的細節。比如\bopin{ㄩ}{\"u}這個介音,我們有時會略去兩點,把它簡寫為\boxpinyin{u},搞得就像\bopin{ㄨ}{u}似的。}{“简洁”的要义之一就是,在不致引起歧义的情况下,略去一些麻烦的细节。比如\bopin{ㄩ}{\"u}这个介音,我们有时会略去两点,把它简写为\boxpinyin{u},搞得就像\bopin{ㄨ}{u}似的。}\par
\content{在漢語拼音中,\textbf{當介音\boxpinyin{\"u}以\boxpinyin{j q x}為聲母,或沒有聲母(零聲母)時,\boxpinyin{\"u}可以簡寫成\boxpinyin{u}而不致有歧義。}比如「居」字,它本應拼作\bopin{ㄐㄩ}{j\={\"u}}。但是,因為聲母\bopin{ㄐ}{j}與介音\bopin{ㄨ}{u}絕無結合的可能,所以我們即便把\bopin{ㄩ}{\"u}寫成\bopin{ㄩ}{u},也不會被人誤會。}{在汉语拼音中,\textbf{当介音\boxpinyin{\"u}以\boxpinyin{j q x}为声母,或没有声母(零声母)时,\boxpinyin{\"u}可以简写成\boxpinyin{u}而不致有歧义。}比如“居”字,它本应拼作\bopin{ㄐㄩ}{j\={\"u}}。但是,因为声母\bopin{ㄐ}{j}与介音\bopin{ㄨ}{u}绝无结合的可能,所以我们即便把\bopin{ㄩ}{\"u}写成\bopin{ㄩ}{u},也不会被人误会。}\par
\content{同理,「元\bopin{ㄩㄢˊ}{y\"u\'an}」當中的\boxpinyin{\"u}即便簡寫成\boxpinyin{u}也不會有歧義,所以在漢語拼音中,我們都把它寫作\bopin{ㄩㄢˊ}{\pinyin{yuan2}}。}{同理,“元\bopin{ㄩㄢˊ}{y\"u\'an}”当中的\boxpinyin{\"u}即便简写成\boxpinyin{u}也不会有歧义,所以在汉语拼音中,我们都把它写作\bopin{ㄩㄢˊ}{\pinyin{yuan2}}。}\par
\content{對於\bopin{ㄝ}{\^e}也是如此。因為介音\bopin{ㄧ}{y/i}和\bopin{ㄩ}{\"u}都不可能與\bopin{ㄜ}{e}結合,所以我們把\boxpinyin{\^e}簡寫成\boxpinyin{e}也不會被人誤會:\bopin{ㄧㄝ}{ie}\bopin{ㄩㄝ}{\"ue}。}{对于\bopin{ㄝ}{\^e}也是如此。因为介音\bopin{ㄧ}{y/i}和\bopin{ㄩ}{\"u}都不可能与\bopin{ㄜ}{e}结合,所以我们把\boxpinyin{\^e}简写成\boxpinyin{e}也不会被人误会:\bopin{ㄧㄝ}{ie}\bopin{ㄩㄝ}{\"ue}。}\par
\subsubsubsection{\content{特殊的\bopin{ㄨㄥ}{ong}}{特殊的\bopin{ㄨㄥ}{ong}}}
\content{注音符號體系有一個很大的缺憾:\boxzhuyin{ㄨㄥ}在不同情況下的發音不一樣;在拼音中,這個差異對應著兩個不同的韻母:\boxpinyin{ueng}和\boxpinyin{ong}。在零聲母時,\boxzhuyin{ㄨㄥ}的拼音是\boxpinyin{ueng};而在有聲母時,\boxzhuyin{ㄨㄥ}的拼音是\boxpinyin{ong}。}{注音符号体系有一个很大的缺憾:\boxzhuyin{ㄨㄥ}在不同情况下的发音不一样;但在拼音中,这个差异对应着两个不同的韵母:\boxpinyin{ueng}和\boxpinyin{ong}。在零声母时,\boxzhuyin{ㄨㄥ}的拼音是\boxpinyin{ueng};而在有声母时,\boxzhuyin{ㄨㄥ}的拼音是\boxpinyin{ong}。}\par
\content{比如,「翁」字是\bopin{ㄨㄥ}{\pinyin{weng1}}而「東」字是\bopin{ㄉㄨㄥ}{\pinyin{dong1}}。這裡的\boxpinyin{ong}不是拼音\boxpinyin{ueng}的簡寫,而是與\bopin{ㄥ}{eng}不同的另一個韻母。在通韻中,它們也是屬於兩個不同韻部的。}{比如,“翁”字是\bopin{ㄨㄥ}{\pinyin{weng1}}而“东”字是\bopin{ㄉㄨㄥ}{\pinyin{dong1}}。这里的\boxpinyin{ong}不是拼音\boxpinyin{ueng}的简写,而是与\bopin{ㄥ}{eng}不同的另一个韵母。在通韵中,它们也是属于两个不同韵部的。}\par
\content{另一個特殊的韻母\bopin{ㄩㄥ}{iong}也不屬於通韻的\boxpinyin{eng}韻部,而是屬於\boxpinyin{ong}韻部,比如「窮」字是\bopin{ㄑㄩㄥˊ}{\pinyin{qiong2}},「用」是\bopin{ㄩㄥˋ}{\pinyin{yong4}}。}{另一个特殊的韵母\bopin{ㄩㄥ}{iong}也不属于通韵的\boxpinyin{eng}韵部,而是属于\boxpinyin{ong}韵部,比如“穷”字是\bopin{ㄑㄩㄥˊ}{\pinyin{qiong2}},“用”是\bopin{ㄩㄥˋ}{\pinyin{yong4}}。}\par
\subsubsection{\content{聲調標在哪?}{声调标在哪?}}
\content{對於漢語拼音的初學者而言,聲調的位置也是一個相當令人頭疼的問題。就拿「怪\boxzhuyin{ㄍㄨㄞˋ}」來說,它的拼音\boxpinyin{guai}在韻母中有三個元音字母\footnote{拼音中的元音字母是\boxpinyin{a o e \^e i u \"u}這七個。至於\boxpinyin{y w},它們不算元音字母。},我們何以知道聲調該標在誰的頭頂上呢?}{对于汉语拼音的初学者而言,声调的位置也是一个相当令人头疼的问题。就拿“怪\boxzhuyin{ㄍㄨㄞˋ}”来说,它的拼音\boxpinyin{guai}在韵母中有三个元音字母\footnote{拼音中的元音字母是\boxpinyin{a o e \^e i u \"u}这七个。至于\boxpinyin{y w},它们不算元音字母。},我们何以知道声调该标在谁的头顶上呢?}\par
\content{在這裡,我想用最簡單的規則描述來幫讀者釐清「聲調標在哪」的問題:}{在这里,我想用最简单的规则描述来帮读者理清“声调标在哪”的问题:}\par
\begin{enumerate}
    \item \content{\textbf{聲調只能標在元音上}\footnote{除非像\boxpinyin{\`ng}這樣真的沒有元音,那就沒辦法了。}。在拼音中,空韻母音節一般都補全了韻母,所以絕大多數音節都是至少有一個韻母成分的——也就至少有一個元音。因此,如果音節成分很簡單,如\bopin{ㄇㄚ}{\pinyin{ma1}}\bopin{ㄓˋ}{\pinyin{zhi4}}\bopin{ㄩˇ}{\pinyin{yu3}},我們就不需要猶豫了,加在唯一的元音上就好!}{\textbf{声调只能标在元音上}\footnote{除非像\boxpinyin{\`ng}这样真的没有元音,那就没办法了。}。在拼音中,空韵母音节一般都补全的韵母,所以绝大多数音节都是至少有一个韵母成分的——也就至少有一个元音。因此,如果音節成分很简单,如\bopin{ㄇㄚ}{\pinyin{ma1}},我们就不需要犹豫了,加在唯一的元音上就好!}
    \item \content{\textbf{韻身比介音優先}。當介音和韻母同時存在時,聲調應該標在韻母上,而不在介音上,比如\bopin{ㄉㄨㄛ}{\pinyin{duo1}}\bopin{ㄌㄩㄝˋ}{\pinyin{lve4}}\bopin{ㄨㄛˇ}{\pinyin{wo3}}。但是,如果只有介音而無韻身,那就標在(作韻母的)介音上,比如\bopin{ㄋㄩˇ}{\pinyin{nv3}}\bopin{ㄧ}{\pinyin{yi1}}。}{\textbf{韵身比介音优先}。当介音和韵母同时存在时,声调应该标在韵母上,而不在介音上,比如\bopin{ㄉㄨㄛ}{\pinyin{duo1}}\bopin{ㄌㄩㄝˋ}{\pinyin{lve4}}\bopin{ㄨㄛˇ}{\pinyin{wo3}}。但是,如果只有介音而无韵身,那就标在(作韵母的)介音上,比如\bopin{ㄋㄩˇ}{\pinyin{nv3}}\bopin{ㄧ}{\pinyin{yi1}}。}
    \item \content{\textbf{韻身的第一個字母優先}。如果我們要把聲調標在韻身上,那麼標在韻身的第一個字母上就可以了!\bopin{ㄍㄨㄞˋ}{\pinyin{guai4}}\bopin{ㄓㄡ}{\pinyin{zhou1}}\bopin{ㄔㄨㄤ}{\pinyin{chuang1}}\newline\bopin{ㄑㄩㄥˊ}{\pinyin{qiong2}}都是如此。}{\textbf{韵身的第一个字母优先}。如果我们要把声调标在韵身上,那么标在韵身的第一个字母上就可以了!\bopin{ㄍㄨㄞˋ}{\pinyin{guai4}}\bopin{ㄓㄡ}{\pinyin{zhou1}}\bopin{ㄔㄨㄤ}{\pinyin{chuang1}}\newline\bopin{ㄑㄩㄥˊ}{\pinyin{qiong2}}都是如此。}
    \item \content{\textbf{韻母簡化後,仍然找元音}。按理說,韻母簡化後,我們仍應在韻身的第一個(元音)字母上標注聲調,比如\bopin{ㄍㄨㄟˋ}{\pinyin{gui4}}和\bopin{ㄐㄧㄡˇ}{\pinyin{jiu3}}。但是,如果像\boxpinyin{un}這樣,韻身中的元音已經被簡化得消失了,我們就只能退而求其次,把聲調標在介音上,比如\bopin{ㄔㄨㄣ}{\pinyin{chun1}}\bopin{ㄅㄧㄥˋ}{\pinyin{bing4}}\bopin{ㄩㄣˊ}{\pinyin{yun2}}。}{\textbf{韵母简化后,仍然找元音}。按理说,韵母简化后,我们仍应在韵身的第一个(元音)字母上标注声调,比如\bopin{ㄍㄨㄟˋ}{\pinyin{gui4}}和\bopin{ㄐㄧㄡˇ}{\pinyin{jiu3}}。但是,如果像\boxpinyin{un}这样,韵身中的元音已经被简化得消失了,我们就只能退而求其次,把聲調标在介音上,比如\bopin{ㄔㄨㄣ}{\pinyin{chun1}}\bopin{ㄅㄧㄥˋ}{\pinyin{bing4}}\bopin{ㄩㄣˊ}{\pinyin{yun2}}。}\par
\end{enumerate}
