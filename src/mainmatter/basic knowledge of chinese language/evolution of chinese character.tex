\newpage
\section{\content{漢字的演變}{汉字的演变}}\label{section:evolution of chinese character}
\content{在本節,我們將了解漢字是如何從原始形態發展到如今之模樣的。我會從甲骨文開始,一路講到「二簡字」。}{在本节,我们将了解汉字是如何从原始形态发展到如今之模样的。我会从甲骨文开始,一路讲到“二简字”。}\par
\content{或許讀者會好奇,早已淘汰的甲骨文和廢棄的二簡字有什麼好學的?的確,本書作為「繁簡字體轉換手冊」並不是要教讀者怎麼寫甲骨文、金文、篆文以及二簡字。但是,現今通行的「繁體字」和「簡體字]作為字形演變的環節之一,須得放在整個漢字演變史中來看,才能令我們更深刻地理解文字是如何產生、如何發展的,其中又有什麼樣的規律。另外,二簡字在字形簡化方面同樣延續了這些規律,且比一簡字做得更加系統,並非沒有可取之處。因此,無論是繁體用戶學習簡體字,還是簡體用戶學習繁體字,本節內容都能讓我們有所收獲。}{或许读者会好奇,早已淘汰的甲骨文和废弃的二简字有什么好学的?的确,本书作为“繁简字体转换手册”并不是要教读者怎么写甲骨文、金文、篆文以及二简字。但是,现今通行的“繁体字”和“简体字”作为字形演变的环节之一,须得放在整个汉字演变史中来看,才能令我们更深刻地理解文字是如何产生、如何发展的,其中又有什么样的规律。另外,二简字在字形简化方面同样延续了这些规律,且比一简字做得更加系统,并非没有可取之处。因此,无论是繁体用户学习简体字,还是简体用户学习繁体字,本节内容都能让我们有所收获。}\par
\content{遺憾的是,筆者並非古漢語及書法方面的專家,所以我講解的內容也往往只能浮於表面,且難免出現錯誤和缺漏。不過我還是會盡力為讀者提供可靠的資訊及參考資料。}{遗憾的是,笔者并非古汉语方面的专家,所以我讲解的内容也往往只能浮于表面,且难免出现错误和缺漏。不过我还是会尽力为读者提供可靠的资讯及参考资料。}\par
\subsection{\content{書寫工具的沿革}{书写工具的沿革}}
\content{最早發現的\textbf{甲骨文}都是用刀之類的鋒利物品刻在「甲骨」之上的。「甲骨」是通稱,「甲」一般指龜甲,「骨」一般指牛骨。它們都不易腐爛,能在地下長期保存,只要能被有心人發現,便可以帶著一個古老時代的謎團重見天日。}{最早发现的\textbf{甲骨文}都是用刀之类的锋利物品刻在“甲骨”之上的。“甲骨”是通称,“甲”一般指龟甲,“骨”一般指牛骨。它们都不易腐烂,能在地址长期保存,只要能被有心人发现,便可以带着一个古老时代的谜团重见天日。}\par
\content{晚清金石學家王懿榮便是這樣的有心人。他偶然地從中藥材「龍骨」上發現了形如文字的符號,並斷定這就是早期的漢文字。在隨後的一百多年裡,歷代學者發掘和釋讀出了越來越多的甲骨文,這些甲骨文內容又與史書互相佐證,為人們還原出了更豐富的殷商歷史面貌。}{晚清金石学家王懿荣便是这样的有心人。他偶然地从中药材“龙骨”上发现了形如文字的符号,并断定这就是早期的汉文字。在随后的一百多年里,历代学者发掘出了越来越多的甲骨文,这些甲骨文内容又与史书互相佐证,为人们还原出了更丰富的殷商历史面貌。}\par
\begin{wrapfigure}{O}{.46\textwidth}
	\begin{subcaptionblock}{.15\textwidth}
		\centering
		\includegraphics[width=\textwidth]{甲骨文/史}
		\caption{\content{史}{史}}
	\end{subcaptionblock}
	\begin{subcaptionblock}{.15\textwidth}
		\centering
		\includegraphics[width=\textwidth]{甲骨文/冊}
		\caption{\content{冊}{册}}
	\end{subcaptionblock}
	\begin{subcaptionblock}{.15\textwidth}
		\centering
		\includegraphics[width=\textwidth]{甲骨文/典}
		\caption{\content{典}{典}}
	\end{subcaptionblock}
	\caption{}
	\label{figure:甲骨文中的「史」「冊」「典」}
\end{wrapfigure}
\content{較近的考古證據則表明,商代人的主要書寫工具並不是甲骨,而是\textbf{毛筆和竹簡}\footnote{參見《了不起的文明現場:跟著一線考古隊長穿越歷史》李零等著。}。我們看\cref*{figure:甲骨文中的「史」「冊」「典」},甲骨文中「史」字描繪的是手拿筆桿寫字的模樣;「冊」字則像一排竹簡串起來的樣子;「典」字更形象,它描繪的是雙手端著竹簡的模樣。}{较近的考古证据则表明,商代人的主要书写工具不是甲骨,而是\textbf{毛笔和竹简}\footnote{参见《了不起的文明现场:跟著一线考古队长穿越历史》李零等著。}。我们看\cref*{figure:甲骨文中的「史」「冊」「典」},甲骨文中“史”字描绘的是手拿笔杆写字的模样;“册”字则像一排竹简串起来的样子;“典”字更形象,它描绘的是双手端着竹简的模样。}\par
\content{毛筆、竹簡要比龜甲、兽骨更易獲取,因此它的用途也就更「日常化」。但可惜的是,竹簡易腐,考古界至今也難以得到三千多年前的有字竹簡,所以罕有直接證據能夠證明當時竹簡的用途如何。}{毛笔、竹简要比龟甲、兽骨更易获取,因此它的用途也就更“日常化”。但可惜的是,竹简易腐,考古界至今也难以得到三千多年前的有字竹简,所以罕有直接证据能够证明当明当时竹简的用途如何。}\par
\content{而在同一時期,青銅器則要比甲骨更難獲取。因此青銅器上的\textbf{金文}就必須用來記錄最重要的事情,萬萬不能寫流水賬。就從出土文物來看,商代甲骨文多用於記錄祭祀、戰爭、狩猎、歷法、天象等,內容繁多,種類多樣;但商朝青銅器上的銘文就顯得惜字如金,大多只是記錄鑄者或其祖先的名稱,僅有廖廖幾字。}{而在同一时期,青铜器则要比甲骨更难获取。因此青铜器上的\textbf{金文}就必须用来记录最重要的事情,万万不能写流水账。就从出土文物来看,商代甲骨文多用于记录祭祀、战争、历法、天象等,内容繁多,种类多样;但商朝青铜器上的铭文就显得惜字如金,大多只是记录铸者或其祖先的名称,仅有廖廖几字。}\par
\content{及至西周,乃至春秋、戰國,青銅器工藝漸臻成熟,產量也越來越高,人們對於金文的使用也不再那樣吝嗇。於是,舉凡王公貴族之事,無論大小,都可以記錄在青銅器上,金文的發展便在周代達到最盛;而殷商甲骨文則隨著商代的覆滅,一並失傳。}{及至西周,乃至春秋、战国,青铜器工艺渐臻成熟,产量也越来越高,人们对于金文的使用也不再那样吝啬。于是,举凡王公贵族之事,无论大小,都可以记录在青铜器上,金文的发展便在周代达到最盛;而殷商甲骨文则随着商代的覆灭,一并失传。}\par
\begin{wrapfigure}{O}{.4\textwidth}
	\centering
	\vspace{-1em}
	\includegraphics[width=.3\textwidth]{孔子詩論竹簡.jpg}
	\caption{\content{《孔子詩論》竹簡片段}{《孔子诗论》竹简片段}}
	\label{figure:《孔子詩論》竹簡片段}
	\footnotesize{\content{圖片來源:維基共享資源}{图片来源:维基共享资源}}
\end{wrapfigure}
\content{春秋、戰國人也用竹簡寫字。\cref*{figure:《孔子詩論》竹簡片段}便是《孔子詩論》的竹簡書,其間所寫的文字應當是大篆的一種。除了竹簡之外,人們也偶爾使用\textbf{絲綢}(古稱「帛」)來書寫——這大概是中國歷史上最早的「紙」了\footnote{「自古書契多編以竹簡,其用縑帛者謂之為紙。」——《後漢書·宦者列傳》}。但絲綢既不像竹簡那樣廉價,又不像青銅器那樣容易保存,所以始終沒有普及開來。東漢蔡倫改進造紙術後,紙張的生產成本大大降低,這才逐漸取代笨重的竹簡,成為兩千多年來的主流書寫工具。}{春秋、战国人也用竹简写字。\cref*{figure:《孔子詩論》竹簡片段}便是《孔子诗论》的竹简书,其间所写的文字应当是大篆的一种。除了竹简之外,人们也偶尔使用\textbf{丝绸}(古称“帛”)来书写——这大概是中国历史上最早的“纸”了\footnote{“自古书契多编以竹简,其用缣帛者谓之为纸。”——《后汉书·宦者列传》}。但丝绸既不像竹简那样廉价,又不像青铜器那样容易保存,所以始终没有普及开来。东汉蔡伦改进造纸术后,纸张的生产成本大大降低,这才逐渐最代笨重的竹简,成为两千多年来的主流书写工具。}\par
\content{戰國時期雖有毛筆,但尚不成熟,很少被人使用。當時的人們在竹簡上寫字,通常是用竹簽點漆,或是用刀篆刻\footnote{或許「篆文」的名稱就是由此而來。}——這兩類書寫工具都算是「硬筆」。戰國末期的秦將蒙恬改進了毛筆,才使毛筆逐漸取代硬筆的地位,成為主流的書寫工具。}{战国时期虽有毛笔,但尚不成熟,很少被人使用。当时的人们在竹简上写字,通常是用竹签点漆,或是用刀篆刻\footnote{或许“篆文”的後世名称就是由此而来。}——这两类书写工具都算是“硬笔”。战国末期的秦将蒙恬改进了毛笔,才使毛笔逐渐取代硬笔的地位,成为主流的书写工具。}\par
\content{毛筆和紙張對漢字演變有著重大意義。毛筆運筆靈活,細節豐富,變化無窮,書寫的多樣性比硬筆高出一個檔次,於是後世各種風格的楷書、行書、草書層出不窮。廉價的紙張對於文人墨客來說,同樣不可或缺。相傳王羲之每日練完書法都在住處附近的池中洗筆,經年累月竟把池水洗黑。若他是用絲綢練字的話,恐怕還沒練成就已傾家蕩產了。}{毛笔和纸张对汉字演变有着重大意义。毛笔运笔灵活,细节丰富,变化无穷,书写的多样性比硬笔高出一个档次,于是后世各种风格的楷书、行书、草书层出不穷。廉价的纸张对于文人墨客来说,同样不可或缺。相传王羲之每日练完书法都在住处附近的池中洗笔,经年累月竟把池水洗黑。若他是用丝绸写字的话,恐怕还没练成就已倾家荡产了。}\par
\content{到了現代,隨著硬筆生產技術的提高和人們對書寫效率的要求,硬筆又占據了主流市場;而紙張的地位則從未被撼動。不過現代人還多了一套「書寫工具」,那就是——鍵盤、熒屏和打印機。}{到了现代,随着硬笔生产技术的提高和人们对书写效率的要求,硬笔又占据了主流市场;而纸张的地位则从未被撼动。不过现代人还多了一套“书写工具”,那就是——键盘、屏幕和打印機。}
\subsection{\content{印刷術的發展}{印刷术的发展}}
