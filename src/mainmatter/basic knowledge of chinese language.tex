\chapter{\content{漢語基礎知識}{汉语基础知识}}\label{chapter:basic knowledge of chinese language}
\content{本章不負責講解任何字的寫法,只是單純羅列一些有關漢語言文字的通識。了解乃至掌握這些知識或許可以幫助你更輕鬆地學習繁簡轉換;但這不是必需的。如果你打算跳過不看本章,那也不會有什麼問題啦。}{本章不负责讲解任何字的写法,只是单纯罗列一些有关汉语言文字的通识。了解乃至掌握这些知识或许可以帮你更轻松地学习繁简转换;但这不是必需的。如果你打算跳过不看本章,那也不会有什么问题啦。}\par
\content{提醒讀者,本書面向的人群是有一定漢語知識的人。一個簡單的標準是:如果你的母語是漢語,並且有國中文化水平,那麼本書是適合你的。相反,如果你初學漢語不久,那麼這本書是不適合你的。}{提醒读者,本书面向的人群是有一定汉语知识的人。一个简单的标准是:如果你的母语是汉语,并且有初中文化水平,那么本书是适合你的。相反,如果你初学汉语不久,那么这本书是不适合你的。}\par
\import{basic knowledge of chinese language/}{bopomofo-pinyin.tex}\newpage
\import{basic knowledge of chinese language/}{evolution of chinese character.tex}\newpage
\import{basic knowledge of chinese language/}{chinese characters on computers.tex}\newpage
\import{basic knowledge of chinese language/}{questions about traditional and simplified chinese.tex}\newpage
