\chapter{\content{參考資料}{参考资料}}\label{appendix:reference}
\begin{itemize}
    \item \content{書籍/出版物}{书籍/出版物}
    \begin{itemize}
        \item \content{【大陸】古漢語常用字字典\ 第5版(王力\ 等)}{【大陆】古汉语常用字字典\ 第5版(王力\ 等)}\\\inlineinfo{ISBN}{978-7-100-11916-0}
        \item \content{【大陸】新華字典\ 第12版}{【大陆】新华字典\ 第12版}\\\inlineinfo{ISBN}{978-7-100-17120-5}
        % \item \content{【大陸】說文解字(湯可敬\ 譯註)}{【大陆】说文解字(汤可敬 译注)}
        \item \content{【大陸】現代漢語詞典\ 第7版}{【大陆】现代汉语词典\ 第7版}\\\inlineinfo{ISBN}{978-7-100-12450-8}
        \item \content{【香港】現代漢語詞典(繁體字版)}{【香港】现代汉语词典(繁体字版)}\\\inlineinfo{ISBN}{978-962-07-0211-2}
        \item \content{【台灣】字型散步(柯志杰、蘇煒翔)}{【台湾】字型散步(柯志杰、苏炜翔)}\\\inlineinfo{ISBN}{978-986-235-401-8}
        \item \content{【大陸】文字學概要(裘錫圭)}{【大陆】文字学概要(裘锡圭)}\\\inlineinfo{ISBN}{978-7-100-09370-5}
        \item \content{【大陸】中國文字學(唐䦨)}{【大陆】中国文字学(唐兰)}\\\inlineinfo{ISBN}{978-7-5325-3903-1}
    \end{itemize}
    \item \content{政府文件}{政府文件}
    \begin{itemize}
        \item \content{【大陸】通用規範漢字表}{【大陆】通用规范汉字表}
        \item \content{【大陸】簡化字總表(1986年版)}{【大陆】简化字总表(1986年版)}
        \item \content{【香港】常用字字形表}{【香港】常用字字形表}
        \item \content{【香港】香港增補字符集}{【香港】香港增补字符集}
        \item \content{【香港】香港電腦漢字參考手冊}{【香港】香港电脑汉字参考手册}
    \end{itemize}
    \item \content{網路資源}{网络资源}
    \begin{itemize}
        \item \content{{【大陸】新一站式繁體字學習手冊(知乎\ liuhl6)}}{【大陆】新一站式繁体字学习手册(知乎\ liuhl6)}\\\inlinelinkinfo{https://zhuanlan.zhihu.com/p/546029604}{zhuanlan.zhihu.com/p/546029604}
        \item \content{MediaWiki內建繁簡轉換表}{MediaWiki内建繁简转换表}\\\inlinelinkinfo{https://phabricator.wikimedia.org/source/mediawiki/browse/master/includes/languages/data/ZhConversion.php}{phabricator.wikimedia.org/source/.../ZhConversion.php}
        \item \content{【台灣】教育部重編國語辭典修訂本}{【台湾】教育部重编国语辞典修订本}\\\inlinelinkinfo{https://dict.revised.moe.edu.tw}{dict.revised.moe.edu.tw}
        \item \content{【台灣】教育部異體字字典(台灣學術網路)}{【台湾】教育部异体字字典(台湾学术网络)}\\\inlinelinkinfo{https://dict.variants.moe.edu.tw}{dict.variants.moe.edu.tw}
        \item \content{【台灣】萌典}{【台湾】萌典}\\\inlinelinkinfo{https://www.moedict.tw}{www.moedict.tw}
        \item \content{甲骨文數據庫}{甲骨文数据库}\\\inlinelinkinfo{https://github.com/Chinese-Traditional-Culture/JiaGuWen}{github.com/Chinese-Traditional-Culture/JiaGuWen}
        \item \content{【台灣】小學堂文字學資料庫}{【台湾】小学堂文字学资料库}\\\inlinelinkinfo{https://xiaoxue.iis.sinica.edu.tw}{xiaoxue.iis.sinica.edu.tw}
        \item \content{【大陸】中國大百科全書\ 第三版網路版}{【大陸】中国大百科全书(第三版网络版)}\\\inlinelinkinfo{https://www.zgbk.com}{www.zgbk.com}
        \item \content{【大陸】中國哲學書電子化計劃}{【大陆】中国哲学书电子化计划}\\\inlinelinkinfo{https://ctext.org}{ctext.org}
        \item \content{維基百科(中文)}{维基百科(中文)}\\\inlinelinkinfo{https://zh.wikipedia.org}{zh.wikipedia.org}
        \item \content{維基百科(英文)}{维基百科(英文)}\\\inlinelinkinfo{https://en.wikipedia.org}{en.wikipedia.org}
        \item \content{維基大典(文言)}{维基大典(文言)}\\\inlinelinkinfo{https://zh-classical.wikipedia.org}{zh-classical.wikipedia.org}
        \item \content{【台灣】國際電腦漢字及異體字知識庫}{【台湾】国际电脑汉字及异体字知识库}\\\inlinelinkinfo{https://chardb.iis.sinica.edu.tw}{chardb.iis.sinica.edu.tw}
        \item \content{字統網}{字统网}\\\inlinelinkinfo{https://zi.tools}{zi.tools}
    \end{itemize}
    \item \content{書中的圖片來源}{书中的图片来源}
    \begin{itemize}
        \item \begin{spacing}{.5}
            \content{維基共享資源}{维基共享资源}\vspace{2pt}\\
            \tiny
            \cref{figure:說文-艸部}
            \cref{figure:章太炎《駁中國用萬國新語說》「紐文」一頁}
            \cref{figure:《孔子詩論》竹簡片段}
            \cref{figure:徒手刻版}
            \cref{figure:九疊篆陽文印「鷹坊之印」}
            \cref{figure:鳥蟲篆陰文印「大千世界」}
            \cref{figure:《曹全碑》片段}
        \end{spacing}
        \item \begin{spacing}{.5}
            \content{小學堂文字學資料庫}{小学堂文字学资料库}\vspace{2pt}\\
            \tiny
            \cref{figure:甲骨文中的「史」「冊」「典」}
            \cref{table:一些漢字的商金文與甲骨文}
            \cref{figure:《說文》小篆中的「月」與「肉」}
            \cref{figure:「隱」字的早期寫法}
            \\\content{另有大量內文用字,無法盡述}{另有大量内文用字,无法尽述}
        \end{spacing}
        \item \begin{spacing}{.5}
            \content{中國哲學書電子化計劃}{中国哲学书电子化计划}\vspace{2pt}\\
            \tiny
            \cref{figure:《干祿字書》中的「隱」}
        \end{spacing}
    \end{itemize}
	\item \content{使用字體}{使用字体}
	\begin{itemize}
		\item \content{思源宋體(繁體中文)}{思源宋体(繁体中文)} Noto Serif CJK TC
		\item \content{思源宋體(簡體中文)}{思源宋体(简体中文)} Noto Serif CJK SC
		% \item \content{思源宋體(香港字型)}{思源宋体(香港字型)} Noto Serif CJK HK
		\item \content{思源黑體(繁體中文)}{思源黑体(繁体中文)} Noto Sans CJK TC
		\item \content{思源黑體(簡體中文)}{思源黑体(简体中文)} Noto Sans CJK SC
		% \item \content{思源黑體(香港字型)}{思源黑体(香港字型)} Noto Sans CJK HK
		\item \content{全宋體}{全宋体} FSung
		% \item \content{自製部分宋体字}{自制部分宋体字}
		\item \content{全字庫正楷體}{全字库正楷体}
		\item \content{楷體\_GB2312}{楷体\_GB2312}
	\end{itemize}
\end{itemize}
