\chapter{\content{漢字演變圖譜}{汉字演变图谱}}\label{appendix:evolution of chinese characters}
\content{本附錄依據裘錫圭《文字學概要》(下稱「《概要》」)、唐䦨《中國文字學》(下稱《字學》)等資料整理而成,先舉出摘要內容,再據此畫出演變圖譜。}{本附录依据裘锡圭《文字学概要》(下称《概要》)、唐兰《中国文字学》(下称《字学》)等资料整理而成,先举出摘要内容,再据此画出演变图谱。}\par
\section{\content{摘要部分}{摘要部分}}
\begingroup
\NewDocumentCommand{\from}{sm}{\IfBooleanTF{#1}{\textcolor{darkgray}{\smash{(#2)}}}{\textcolor{darkgray}{(#2)}}}
\small
\setlist{nosep}
\setlist[1]{leftmargin=1.5em}
\setlist[2]{leftmargin=1.5em,label=+}
\begin{itemize}
	\item \content{商代}{商代}
	\begin{itemize}
		\item \content{我們可以把甲骨文看作當時的一種比較特殊的俗體字,而金文大體上可以看作當時的正體字。\from{《概要》48頁}}{我们可以把甲骨文看作当时的一种比较特殊的俗体字,而金文大体上可以看作当时的正体字。\from{《概要》48页}}
	\end{itemize}
	\item \content{西周春秋時代}{西周春秋时代}
	\begin{itemize}
		\item \content{西周春秋時代一般金文的字體,大概可以代表當時的正體。一部分寫得比較草率的金文,則反映了俗體的一些情況。\from{《概要》53頁}}{西周春秋时代一般金文的字体,大概可以代表当时的正体。一部分写得比较草率的金文,则反映了俗体的一些情况。\from{《概要》53页}}
	\end{itemize}
	\item \content{戰國至西漢初年}{战国至西汉初年}
	\begin{itemize}
		\item \content{在春秋時代的各個主要國家中,建立在宗周故地的秦國,是最忠實地繼承了西周王朝所使用的文字的傳統的國家……而且戰國時代秦國文字的正體後來演變為小篆,俗體則發展成為隸書,俗體雖然不是對正體沒有影響,但是始終沒有打亂正體的系統。\from{《概要》58頁}}{在春秋时代的各个主要国家中,建立在宗周故地的秦国,是最忠实地继承了西周王朝所使用的文字的传统的国家……而且战国时代秦国文字的正体后来演变为小篆,俗体则发展成为隶书,俗体虽然不是对正体没有影响,但是始终没有打乱正体的系统。\from{《概要》58页}}
		\item \content{各國的正體都跟西周春秋文字比較接近,因此它們之問的共同性也比較多。不過在我們看到的六國文字資料裡,幾乎找不到一種沒有受到俗體的明顯影響的資料。\from{《概要》64頁}}{各国的正体都跟西周春秋文字比较接近,因此它们之间的共同性也比较多。不过在我们看到的六国文字资料里,几乎找不到一种没有受到俗体的明显影响的资料。\from{《概要》64页}}
		\item \content{東方各國俗體的字形跟傳統的正體的差別往往很大,而且由於俗體使用得非常廣泛,傳統的正體已經被衝擊得潰不成軍了……戰國時代東方各國通行的文字,跟西周晚期和春秋時代的傳統的正體相比,幾乎已經面目全非。\from{《概要》58頁}}{东方各国俗体的字形跟传统的正体的差别往往很大,而且由于俗体使用得非常广泛,传统的正体已经被冲击得溃不成军了……战国时代东方各国通行的文字,跟西周晚期和春秋时代的传统的正体相比,几乎已经面目全非。\from{《概要》58页}}
		\item \content{春秋戰國時代的秦國文字是逐漸演變為小篆的,小篆跟統一前的秦國文字之間並不存在截然分明的界線。\from{《概要》71頁}}{春秋战国时代的秦国文字是逐渐演变为小篆的,小篆跟统一前的秦国文字之间并不存在截然分明的界线。\from{《概要》71页}}
		\item \content{從考古發現的秦系文字資料看,戰國晚期是隸書形成的時期……秦國人在日常使用文字的時候,為了書寫的方便也在不斷破壞、改造正體的字形。由此產生的秦國文字的俗體,就是隸書形成的基礎。\from{《概要》74頁}}{从考古发现的秦系文字资料来看,战国晚期是隶书形成的时期……秦国人在日常使用文字的时候,为了书写的方便也在不断破坏、改造正体的字形。由此产生的秦国文字的俗体,就是隶书形成的基础。\from{《概要》74页}}
		\item \content{六國文字的日漸草率,正是隸書的先導。秦朝用小篆來統一文字,但是民間……寫新朝的文字時,把很莊重的小篆,四平八穩的結構打破了。這種通俗的,變了面目的,草率的寫法,最初只通行於下層社會,統治階級因為他們是賤民……把它們叫做“隸書”,徒隸的書。\from{《字學》131頁}}{六国文字的日渐草率,正是隶书的先导。秦朝用小篆来统一文字,但是民间……写新朝的文字时,把很庄重的小篆,四平八稳的结构打破了。这种通俗的,变了面目的,草率的写法,最初只通行于下层社会,统治阶级因为他们是贱民……把它们叫做“隶书”,徒隶的书。\from{《字学》131页}}
		\item \content{我們雖然在原則上不同意隸書有一部分是承襲六國文字的說法,卻並不否定隸書所從出的篆文或篆文俗體以至隸書本身,曾受到東方國家文字的某些影響的可能性。即使把為秦所統一的東方國家的人民寫篆文和隸書時,有時使用本國舊字形的情況排除在外,仍然可以找到這種影響的某些跡象。\from{《概要》78頁}}{我们虽然在原则上不同意隶书有一部分是承袭六国文字的说法,却并不否定隶书所从出的篆文或篆文俗体以至隶书本身,曾受到东方国家文字的某些影响的可能性。即使把为秦所统一的东方国家的人民写篆文和隶书时,有时使用本国旧字形的情况排除在外,仍然可以找到这种影响的某些迹象。\from{《概要》78页}}
		\item \content{總之,隸書是上層統治階段\footnote{原文或有筆誤,疑作「階級」。}所看不起的。秦代統治者允許官府用隸書來處理日常事務,是迫於形勢不得不然,並不說明他們喜歓或重視這種字體。在比較莊重皫場合,一般是不用隸書的。\from{《概要》79頁}}{总之,隶书是上层统治阶段\footnote{原文或有笔误,疑作“阶级”。}所看不起的。秦代统治者允许官府用隶书来处理日常事务,是迫于形势不得不然,并不说明他们喜欢或重视这种字体。在比较庄重的场合,一般是不用隶书的。\from{《概要》79页}}
		\item \content{秦代和西漢早期的隸書是尚未成熟的早期隸書。西漢武帝時代可以看作隸書由不成熟發展到成熟的時期。\from{《概要》83頁}}{秦代和西汉早期的隶书是尚未成熟的早期隶书。西汉武帝时代可以看作隶书由不成熟发展到成熟的时期。\from{《概要》83页}}
		\item \content{習慣上把具備這些特點的隸書稱為漢隸,漢隸形成之前的隸書稱為秦隸。秦隸也稱古隸。由於秦隸也包括漢隸形成前的漢代隸書,稱它為古隸比較合理。\from{《概要》85頁}}{习惯上把具备这些特点的隶书称为汉隶,汉隶形成之前的隶书称为秦隶。秦隶也称古隶。由于秦隶也包括汉隶形成前的汉代隶书,称它为古隶比较合理。\from{《概要》85页}}
	\end{itemize}
	\item \content{西漢至東漢初年}{西汉至东汉初年}
	\begin{itemize}
		\item \content{隸書是草率的篆書,草書又是草率的隸書,六國末年就產生了隸書,漢初就產生了草書。\from{《字學》96頁}}{隶书是草率的篆书,草书又是草率的隶书,六国末年就产生了隶书,汉初就产生了草书。\from{《字学》96页}}
		\item \content{到了漢代,隸書取代小篆成為主要字體,漢字發展史就脫離古文字階段而進入隸楷階段了,漢代以後,小篆就成為主要用來刻印章、銘金石的古字體。\from{《概要》73頁}}{到了汉代,隶书取代小篆成为主要字体,汉字发展史就脱离古文字阶段而进入隶楷阶段了,汉代以后,小篆成为主要用来刻印章、铭金石的古字体。\from{《概要》73页}}
		\item \content{草書\footnote{原文此處是指章草。}的形成比八分\footnote{原文此處是指漢隸。關於「八分」這個概念,坊間說法不一。有說是指隸書(裘錫圭是這一派),有說是指楷書(唐䦨是這一派)。引者認為,原文要表達的意思確實是漢隸;即便最後人們證明了「八分」是指草書,原作者也會把此處的「八分」改成「漢隸」。}稍晚一些。不過,作為草書形成基礎的草率的隸書俗體,有很大一部分在古隸階段就已經存在。所以也可以說,八分和章草是分別由古隸的正體和俗體發展而成的……草書字形不出自成熟的隸書而是出自古隸的例子是常見的,不但是偏旁的寫法,就是整字的寫法也往往由古隸變來。\from{《概要》92頁}}{草书\footnote{原文此处是指章草。}的形成比八分\footnote{原文此处是指汉隶。关于“八分”这个概念,坊间说法不一。有说是指隶书(裘锡圭是这一派),有说是指楷书(唐兰是这一派)。引者认为,原文要表达的意思确实是汉隶;即便最后人们证明了“八分”是指楷书,原作者也会把此处的“八分”改为“汉隶”。}稍晚一些。不过,作为草书形成基础的草率的隶书俗体,有很大一部分在古隶阶段就已经存在。所以也可以说,八分和章草是分别由古隶的正体和俗体发展而成的……草书字形不出自成熟的隶书而是出自古隶的例子是常见的,不但是偏旁的写法,就是整字的写法也往往由古隶变来。\from{《概要》92页}}
		\item \content{草書的形成至遲不會晚於元、成之際,很可能在宣、元時代就已經形成了。\from{《概要》91頁}}{草书的形成至迟不会晚于元、成之际,很可能在宣、元时代就已经形成了。\from{《概要》91页}}
		\item \content{所以草書成熟,至晚也在西漢末,東漢初。它大概是逐漸演化成的,所以要找一個真正的創始時期,很不容易。\from{《字學》138頁}}{所以草书成熟,至晚也在西汉末,东汉初。它大概是逐渐演化成的,所以要找一个真正的创始时期,很不容易。\from{《字学》138页}}
	\end{itemize}
	\item \content{東漢至魏晉}{东汉至魏晋}
	\begin{itemize}
		\item \content{隸書又產生了今隸\footnote{今隸是楷書的別稱。},草書又產生了行書,這是現在還通行的書寫技術。\from{《字學》96頁}}{隶书又产生了今隶\footnote{今隶是楷书的别称。},草书又产生了行书,这是现在还通行的书写技术。\from{《字学》96页}}
		\item \content{大約在東漢中期,從日常使用的隸書裡演變出了一種跟古隸和八分有明顯區別的比較簡便的俗體……一般人日常所用的隸書卻大都已經是這種俗體了……為了區別於正規的隸書,我們姑且把這種字體稱為新隸體。\from{《概要》95頁}}{大约在东汉中期,从日常使用的隶书里演变出了一种跟古隶和八分有明显区别的比较简便的俗体……一般人日常所用的隶书却大都已经是这种俗体了……为了区别于正规的隶书,我们姑且把这种字体称为新隶体。\from{《概要》95页}}
		\item \content{在東漢晚期還出現了一種新的字體,就是行書。\from{《概要》95頁}}{在东汉晚期还出现了一种新的字体,就是行书。\from{《概要》95页}}
		\item \content{早期行書雖然不是新隸體的一種草體,畢竟是在帶有較多草書筆意的新隸體的基礎上發展出來的一種字體,它跟草率的新隸體不可避免地會有一些相似的特色……因此,要在早期行書跟草率的新隸體之間劃出一條很明確的界線,也是有困難的。\from{《概要》97頁}}{早期行书虽然并不是新隶体的一种草体,毕竟是在带有较多草书笔意的新隶体的基础上发展出来的一种字体,它跟草率的新隶体不可避免地会有一些相似的特色……因此,要在早期行书跟草率的新隶体之间划出一条很明确的界线,也是有困难的。\from{《概要》97页}}
		\item \content{所能看到的最古的楷書是鍾繇所寫的宣示表等帖的臨摹本的刻本。宣示表等帖的字體顯然是脫胎於早期行書的。如果把規整一派的早期行書寫得端莊一點……就會形成宣示表那種字體……以上所說如果基本符合事實的話,我們簡直可以把早期的楷書看作早期行書的一個分支。\from{《概要》97頁}}{所能看到的最古的楷书是钟繇所写的宣示表等帖的临摹本的刻本。宣示表等帖的字体显然是脱胎于早期行书的。如果把规整一派的早期行书写得端庄一点……就会形成宣示表那种字体……以上所说如果基本符合事实的话,我们简直可以把早期的楷书看作早期行书的一个分支。\from{《概要》97页}}
		\item[-] \content{字法之變至隸極矣,然猶有古焉,至楷法則無古矣。在漢建初,有王次仲者,始以隸字作楷法。所謂楷法者,今之正書也。\footnote{這種說法與《概要》有出入。歷代書法家皆以王次仲為楷書鼻祖,但今人沒有發現任何與之相關的考古學證據,無法斷定這種說法是不是以訛傳訛。至於王次仲本人,其生卒年缺少記載,生平也不詳,僅有的關於他的故事也無異於神話傳說,實在不足信。}\from{《宣和書譜·卷三》}}{字法之变至隶极矣,然犹有古焉,至楷法则无古矣。在汉建初,有王次仲者,始以隶字作楷法。所谓楷法者,今之正书也。\footnote{这种说法与《概要》有出入。历代书法家皆以王次仲为楷书鼻祖,但今人没有发现任何与之相关的考古学证据,无法断定这种说法是不是以讹传讹。至于王次仲本人,其生卒年缺少记载,生平也不详,仅有的关于他的故事也无异于神话传说,实在不足信。}\from{《宣和书谱·卷三》}}
		\item \content{漢世本只有隸書的楷法,跟草書的篇章,到漢末才有劉德昇的行書。行書本是楷隸的簡別字,所以容易流行。到鍾繇,除了用行書來寫相聞書,用楷法來寫碑刻外,寫八分時,卻參入了行書的體勢,成為一種新體,那就是章程書\footnote{按唐䦨的說法,章程書就是後人所稱的「正書」或「真書」。}了……這種新體,依然是由隸書演化下來的。\from{《字學》143頁}}{汉世本只有隶书的楷法,跟草书的篇章,到汉末才有刘德昇的行书。行书本是楷隶的简别字,所以容易流行。到钟繇,除了用行书来写相闻书,用楷法来写碑刻外,写八分时,却参入了行书的体势,成为一种新体,那就是章程书\footnote{按唐兰的说法,章程书就是后人所称的“正书”或“真书”。}了……这种新体,依然是由隶书演化下来的。\from{《字学》143页}}
	\end{itemize}
	\item \content{南北朝至隋唐}{南北朝至隋唐}
	\begin{itemize}
		\item \content{東晉時代的有些新隸體,跟行、楷已經相當接近。到了南北朝,就出現了在鍾王楷書的影響下由新隸體演變而成的一種楷書……在北朝的碑志裡,這種楷書較長期地占據著統治地位。由於使用這種楷書的北魏碑志數量很多,後人稱這種楷書為魏碑體……唐以後,魏碑體基本上退出了歷史舞台。\from{《概要》99頁}}{东晋时代的有些新隶体,跟行、楷已经相当接近。到了南北朝,就出现了在钟王楷书的影响下由新隶体演变而成的一种楷书……在北朝的碑志里,这种楷书较长期地占据着统治地位。由于使用这种楷书的北魏碑志数量很多,后人称这种楷书为魏碑体……唐以后,魏碑体基本上退出了历史舞台。\from{《概要》99页}}
		\item \content{鍾王楷書脫胎於行書,作為碑刻上的正體來用,結体和筆法都有不夠莊嚴穩重的地方。南北朝時人已經為弓這個原因,對鍾王楷書作了一些改造。不過直到唐初的歐陽詢,才較好地完成了這項改造工作。因此也有人認為楷書到唐初才真正成熟。\from{《概要》99頁}}{钟王楷书脱胎于行书,作为碑刻上的正体来用,结体和笔法都有不够庄严稳重的地方。南北朝时人已经为了这个原因,对钟王楷书作了一些改造。不过直到唐初的欧阳询,才较好地完成了这项改造工作。因此也有人认为楷书到唐初才真正成熟。\from{《概要》99页}}
		\item \content{楷書雖也把隸書的體勢,變成新體,究竟比真書古雅……阮元有北碑南帖之說,不知一種是楷書,一種是真書,本非同體。南朝的碑志,同樣也是楷書。歐陽詢也還是楷書的系統,一直到虞、禇用真書,唐太宗用行書來寫石刻,這種界限才打破。\from{《字學》144頁}}{楷书虽也把隶书的体势,变成新体,究竟比真书古雅……阮元有北碑南帖之说,不知一种是楷书,一种是真书,本非同体。南朝的碑志,同样也是楷书。欧阳询也还是楷书的系统,一直到虞、褚用真书,唐太宗用行书来写石刻,这种界限才打破。\from{《字学》144页}}
		\item \content{在魏晉時代,由於早期行書和楷書的書法的影響,章草逐漸演變成為今草……有些字在今草裡既有來自章草的寫法,也有楷書草化的寫法……總之,要比章草更「草」,因此也比章草更不易辨認。使用今草的人範圍很窄,主要是一些文人學士。\from{《概要》100頁}}{在魏晋时代,由于早期行书和楷书的书法的影响,章草逐渐演变成为今草……有些字在今草里既有来自章草的写法,也有楷书草化的写法……总之,要比章草更“草”,因此也比章草更不易辨认。使用今草的人范围很窄,主要是一些文人学士。\from{《概要》100页}}
		\item \content{隨著楷書的發展和今草的形成,行書也相應地演變成為介於楷書和今草之間的一種字體,面貌跟早期行書有了明顯的不同。\from{《概要》101頁}}{随着楷书的发展和今草的形成,行书也相应地演变成为介于楷书和今草之间的一种字体,面貌跟早期行书有了明显的不同。\from{《概要》101页}}
		\item \content{草書到了唐以後,又出新體,那是張旭的狂草,寫出來別人多不能識,就完全變成藝術。\from{《字學》141頁}}{草书到了唐以后,又出新体,那是张旭的狂草,写出来别人多不能识,就完全变成艺术。\from{《字学》141页}}
	\end{itemize}
\end{itemize}
\endgroup
